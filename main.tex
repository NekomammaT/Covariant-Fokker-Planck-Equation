% !TEX TS-program = xelatex%

\documentclass[aps, prd
, preprint
%, twocolumn
, nofootinbib 
, notitlepage
%, superscriptaddress
, longbibliography
]{revtex4-1}
\usepackage{graphicx}
\usepackage[caption=false]{subfig}
\usepackage{mathrsfs}
\usepackage{amsmath,amssymb}
\usepackage{bm}
\usepackage{braket}
\usepackage{listings}
\usepackage{cases}
\usepackage{comment}
\usepackage{soul}
\usepackage{cancel}
\usepackage{cases}
\usepackage[utf8]{inputenc}
\usepackage{url}
\usepackage{longtable}
\usepackage[normalem]{ulem}
\usepackage[colorlinks=true
,urlcolor=blue
,anchorcolor=blue
,citecolor=blue
,filecolor=blue
,linkcolor=blue
,menucolor=blue
%,pagecolor=blue
,linktocpage=true
,pdfproducer=medialab
,pdfa=true
]{hyperref}

%\usepackage{mathpazo}
%\usepackage[no-math]{fontspec}
%\setmainfont{Palatino}
%\setsansfont{Optima}

\newcommand{\dif}[2]{\frac{\mathrm{d} #1}{\mathrm{d} #2}}
\newcommand{\pdif}[2]{\frac{\partial #1}{\partial #2}}
\newcommand{\ppdpdpd}[3]{\frac{\partial^2 #1}{\partial #2\partial #3}}
\newcommand{\pppdpdpdpd}[4]{\frac{\partial^3 #1}{\partial #2\partial #3\partial #4}}
\newcommand{\var}[2]{\frac{\delta #1}{\delta #2}}
\newcommand{\dd}{\mathrm{d}}
\newcommand{\DD}{\mathscr{D}}
\newcommand{\ee}{\mathrm{e}}
\newcommand{\diag}{\mathrm{diag}}
\newcommand{\sgn}{\mathrm{sgn}}
\newcommand{\Mpl}{M_\text{Pl}}
\newcommand{\ns}{n_{{}_\mathrm{S}}}
\newcommand{\cs}{c_{{}_\mathrm{S}}}
\newcommand{\IR}{\text{IR}}
\newcommand{\UV}{\text{UV}}
\renewcommand{\Re}{\mathrm{Re}}
\renewcommand{\Im}{\mathrm{Im}}
\newcommand{\dk}{\frac{\dd^3k}{(2\pi)^3}}
\newcommand{\bbalpha}{{\alpha\!\!\!\alpha}}
\newcommand{\dps}{\displaystyle}
\newcommand{\SIA}{S_\text{IA}}
\newcommand{\eff}{\text{eff}}
\newcommand{\kdx}{\mathbf{k}\cdot\mathbf{x}}
\newcommand{\Det}{\mathrm{Det}}
\newcommand{\Tr}{\mathrm{Tr}}
\newcommand{\IA}{\mathrm{IA}}

\newcommand{\calD}{\mathcal{D}}
\newcommand{\scrD}{\mathscr{D}}
\newcommand{\calg}{\mathcal{g}}
\newcommand{\calH}{\mathcal{H}}
\newcommand{\scrH}{\mathscr{H}}
\newcommand{\uI}{\text{I}}
\newcommand{\calJ}{\mathcal{J}}
\newcommand{\scrJ}{\mathscr{J}}
\newcommand{\calL}{\mathcal{L}}
\newcommand{\scrL}{\mathscr{L}}
\newcommand{\calN}{\mathcal{N}}
\newcommand{\calO}{\mathcal{O}}
\newcommand{\scrO}{\mathscr{O}}
\newcommand{\calP}{\mathcal{P}}
\newcommand{\calR}{\mathcal{R}}
\newcommand{\uR}{\text{R}}

\newcommand{\dphi}{\delta\phi}
\newcommand{\dpi}{\delta\pi}
\newcommand{\Dpi}{\Delta\pi}

\newcommand{\bae}[1]{\begin{align} #1 \end{align}}
\newcommand{\bce}[1]{\begin{cases} #1 \end{cases}}
\newcommand{\bfe}[4]{
\begin{figure} 
	\centering
	\includegraphics[#1]{#2}
	\caption{#3}
	\label{#4}
\end{figure}}
\newcommand{\bpme}[1]{\begin{pmatrix} #1 \end{pmatrix}}

\newcommand{\Red}[1]{\textcolor{red}{\sffamily #1}}
\newcommand{\Mag}[1]{\textcolor{magenta}{\sffamily #1}}
\newcommand{\Blue}[1]{\textcolor{blue}{\sffamily #1}}
\newcommand{\mathblue}[1]{\textcolor{blue}{#1}}
\newcommand{\YT}[1]{\textcolor{blue}{\sffamily [YT: #1]}}

\allowdisplaybreaks[3]



\begin{document}
\title{Covariant Fokker-Planck Equation}
\date{\today}

\author{Yuichiro Tada}
\email{tada.yuichiro@e.mbox.nagoya-u.ac.jp}
\affiliation{Department of Physics, Nagoya University, Nagoya 464-8602, Japan}
%\affiliation{Institut d'Astrophysique de Paris, GReCO, UMR 7095 du CNRS et Sorbonne Universit\'e, 
%98bis boulevard Arago, 75014 Paris, France}


\begin{abstract}
Recently Kitamoto pointed out that the one-point quadratic UV terms caused by non-parallel transportation of the vielbein is non-negligible as a new source of the drift term in the Langevin equation,
resulting in a covariant Fokker-Planck equation in the non-linear sigma model~\cite{Kitamoto:2018dek}.
We are trying to summarize all our knowledge, respecting this article.
\begin{table}[htbp]
	\centering
	\caption{Notation}
	\begin{tabular}{rl}
		$\nabla$: & field-space covariant derivative \\
		$D$: & phase-space covariant derivative \\
		$\calD$: & Graham's covariant derivative
	\end{tabular}
\end{table}
\end{abstract}

\maketitle
%\tableofcontents


\section{Field-space formulation}

\subsection{spin connection}

According to Ref.~\cite{Kitamoto:2018dek}, the local-frame fields projected by the vielbein seem
important rather than the original fields themselves. Therefore let us first derive EoM in terms of the local frame.
The original Hamiltonian equations without metric perturbations are given by
\bae{
    \bce{
        \dps
        \partial_N\phi^I=G^{IJ}\frac{\pi_J}{H}, \\
        \dps
        \partial_N\pi_I=-3\pi_I-\frac{V_I}{H}+\frac{1}{a^2H}\partial_i\left(G_{IJ}\partial_i\phi^J\right)
        -\frac{1}{2a^2H}G_{JK,I}\partial_i\phi^J\partial_i\phi^K-\frac{1}{2H}G^{JK}{}_{,I}\pi_J\pi_K.
    }
}
They can be summarized in a covariant way as
\bae{
    \partial_N\phi^I=G^{IJ}\frac{\pi_J}{H}, \qquad D_N\pi_I=-3\pi_I-\frac{V_I}{H}+\frac{1}{a^2H}D_i\left(G_{IJ}\partial_i\phi^J\right),
}
with use of the covariant spacetime derivatives,
\bae{
    D_N\pi_I=\partial_N\pi_I-\Gamma^K_{IJ}\pi_K\partial_N\phi^J, \qquad 
    D_i\calO_I=\partial_i\calO_I-\Gamma^K_{IJ}\calO_K\partial_i\phi^J.
} 

\newpage

For a function only of $\phi$, these derivatives can be clearly rewritten by the covariant field derivative $\nabla_I$ as
\bae{
    D_{N,i}\calO(\phi)=(\partial_{N,i}\phi^I)\nabla_I\calO(\phi).
}
Therefore, introducing the spin connection $\omega_{Iab}$ by
\bae{
    \omega_{Iab}&=e^J_a\nabla_Ie_{Jb} \nonumber \\
    &=\frac{1}{2}e^J_a\left(\partial_Ie_{Jb}-\partial_Je_{Ib}\right)
    -\frac{1}{2}e^I_b\left(\partial_Ie_{Ja}-\partial_Je_{Ia}\right)
    -\frac{1}{2}e^J_ae^K_b(\partial_Je_{Kc}-\partial_Ke_{Jc})e_{Ic},
}
the Hamiltonian equations can be projected to the local frame:
\begin{subnumcases}{}
    \dps
    (\partial_N\phi)_a=\frac{\pi_a}{H}, \\
    \dps
    \partial_N\pi_a=-3\pi_a-\frac{V_a}{H}+\frac{1}{a^2H}\partial_i(\partial_i\phi)_a
    -\frac{1}{H}e^J_c\omega_{Jab}\left(\pi_b\pi_c-\frac{1}{a^2}(\partial_i\phi)_b(\partial_i\phi)_c\right),
    \label{eq: pia EoM}
\end{subnumcases}
where\footnote{Note that field-space vectors are not $\phi^I$ itself but its derivatives $\partial_{N,i}\phi^I$ and that is why the $\phi$'s projection is defined in the way of Eq.~(\ref{eq: def of projection}), that is, $(\partial_{N,i}\phi)_a=e_{Ia}\partial_{N,i}\phi^I$ is different from $\partial_{N,i}(e_{Ia}\phi^I)$.}
\bae{\label{eq: def of projection}
    (\partial_{N,i}\phi)_a=e_{Ia}\partial_{N,i}\phi^I, \qquad \pi_a=e^I_a\pi_I, \qquad V_a=e^I_aV_I.
}

While the third term in the right side of Eq.~(\ref{eq: pia EoM}) can be omitted in the long-wavelength limit, the last term gives a non-negligible contribution, according to Ref.~\cite{Kitamoto:2018dek}, 
and such a one-point correlator is estimated as
\bae{
    \pi_a\pi_b-\frac{1}{a^2}(\partial_i\phi)_a(\partial_i\phi)_b\simeq \frac{3}{2}H^2g_{ac}g_{bc}=:\frac{3}{2}H^2g^2_{ab}.
}
Here $g_{ab}$ represents the mode function of the local-frame UV fields. 
The power spectrum of the original fields can be expressed by this one as 
$\calP_{\phi^I\phi^J}=g_{ab}e^I_bg_{ac}e^J_c=g^2_{bc}e^I_be^J_c$. 
If the mode mixing due to the effective mass matrix can be neglected, i.e. $g^2_{ab}\simeq\left(\frac{H}{2\pi}\right)^2\delta_{ab}$, the power spectrum reduces to the standard one $\calP_{\phi^I\phi^J}\simeq\left(\frac{H}{2\pi}\right)^2G^{IJ}$.
Including the contribution of this term, the slow-roll field-space Langevin eq. for IR fields $\varphi^I$ then reads
\bae{\label{eq: Kitamoto Langevin}
    \dd\varphi^I=-\left(\frac{V^I}{3H^2}+\frac{1}{2}g^2_{ab}e^J_b\nabla_Je^I_a\right)\dd N
    +g_{ab}e^I_b\circ\dd W_a.
}
Note the following identical deformation.
\bae{
    e^I_ag^2_{bc}e^J_c\omega_{Jab}=e^I_ag^2_{bc}e^J_ce^K_a\nabla_Je_{Kb}=g^2_{bc}e^J_cG^{IK}\nabla_Je_{Kb}=g^2_{bc}e^J_c\nabla_Je^I_b.
}

This new drift term represents non-uniformity of the local frame. If the field space is exactly flat,
the vielbein can be taken so that its covariant derivative vanishes at all field space points.
Also if one considers only deterministic background trajectory $\phi^I_0$ without stochastic noise,
there is a useful choice of the vielbein satisfying the parallel transported condition $\dot{\phi}_0^I\nabla_Ie^J_a=0$. 
However in the stochastic cases, one must consider all field space points in principle and therefore this spin connection should be taken into account in general.


\subsection{It\^o-Stratonovich stuff and covariant Fokker-Planck}

In the Langevin eq.~(\ref{eq: Kitamoto Langevin}), we adopted the Stratonovich noise integral, reflecting the fact that the Stratonovich works for metric type interactions.
On the other hand, one has to keep in mind that the potential type interaction should be treated in the It\^o's way.
The potential interaction, or the effective mass correction in other words, affects the local-frame amplitude $g_{ab}$.
Following the Tokuda's approach, one then introduces dummy fields $\bar{\varphi}^I$ representing slightly past field values (or fields with a slightly large coarse-graining scale) and $g_{ab}$ is assumed to be a function of $\bar{\varphi}^I$.
After moving to the It\^o's expression, noting that functions of $\bar{\varphi}^I$ are not affected by differentiations in $\varphi^I$, one can take a limit of $\bar{\varphi}^I\to\varphi^I$:
\bae{\label{eq: Ito in field space}
    \dd\varphi^I&=-\left(\frac{V^I}{3H^2}+\frac{1}{2}g^2_{ab}(\bar{\varphi})e^J_b\nabla_Je^I_a\right)\dd N
    +g_{ab}(\bar{\varphi})e^I_b\circ\dd W_a \nonumber \\
    &=-\left(\frac{V^I}{3H^2}+\frac{1}{2}g^2_{ab}(\bar{\varphi})e^J_b\nabla_Je^I_a-\frac{1}{2}g_{ab}(\bar{\varphi})e^J_b\partial_J(g_{ac}(\bar{\varphi})e^I_c)\right)\dd N+g_{ab}(\bar{\varphi})e^I_b\dd W_a \nonumber \\
    &\hspace{-5pt}\underset{\bar{\varphi}\to\varphi}{=}-\left(\frac{V^I}{3H^2}+\frac{1}{2}g^2_{ab}e^J_b(\nabla_J-\partial_J)e^I_a\right)\dd N+g_{ab}e^I_b\dd W_a \nonumber \\
    &=-\left(\frac{V^I}{3H^2}+\frac{1}{2}\Gamma^I_{JK}\calP_{\phi^J\phi^K}\right)\dd N+g_{ab}e^I_b\dd W_a.
}
Note that the induced term $\frac{1}{2}\Gamma^I_{JK}\calP_{\phi^J\phi^K}\dd N$ is nothing but the correction term introduced by Graham~\cite{graham1985covariant}.
According to this reference, the covariant derivative for the It\^o process $\varphi^I$ is defined by
\bae{
	\calD\varphi^I:=\dd\varphi^I+\frac{1}{2}\Gamma^I_{JK}\calP_{\phi^J\phi^K}\dd N,
}
and then the Langevin equation can be rewritten covariantly making use of this derivative:
\bae{
	\calD\varphi^I=-\frac{V^I}{3H^2}\dd N+g^I_a\dd W_a,
}
where $g^I_a=g_{ab}e^I_b$.

The corresponding Fokker-Planck equation reads
\bae{
    \partial_NP&=\partial_I\left[\left(\frac{V^I}{3H^2}+\frac{1}{2}\Gamma^I_{JK}\calP_{\phi^J\phi^K}\right)P\right]+\frac{1}{2}\partial_I\partial_J(\calP_{\phi^I\phi^J}P) \nonumber \\
    &=\partial_I\left[\left(\frac{V^I}{3H^2}+\frac{1}{2}g_a^J\nabla_Jg^I_a\right)P\right]+\frac{1}{2}\partial_I\left[g^I_a\partial_J(g^J_aP)\right],
}
The formula $\frac{1}{\sqrt{G}}\partial_I(\sqrt{G}\calO^I)=\nabla_I\calO^I$ can enable us to rewrite it covariantly as
\bae{\label{eq: covFP in field space}
    \partial_NP_s&=\nabla_I\left[\left(\frac{V^I}{3H^2}+\frac{1}{2}g^J_a\nabla_Jg^I_a\right)P_s\right]+\frac{1}{2}\nabla_I\left[g^I_a\nabla_J(g^J_aP_s)\right] \nonumber \\
    &=\nabla_I\left(\frac{V^I}{3H^2}P_s\right)+\frac{1}{2}\nabla_I\nabla_J(\calP_{\phi^I\phi^J}P_s),
}
for the field-space scalar PDF $P_s=P/\sqrt{G}$. Therefore the resultant Fokker-Planck equation is indeed 
successfully field-space covariant and independent of the choice of the vielbein.




\section{Phase-Space Formulation}

\subsection{Covariant Fokker-Planck and Langevin equations}

The covariant Fokker-Planck (FP) equation~(\ref{eq: covFP in field space}) in the field-space formulation under the slow-roll approximation
would be naturally generalized to the full phase-space formulation.
In this subsection, let us start by assuming the covariant FP equation in phase space and then derive the corresponding Langevin equation.

Making use of the phase-space covariant field derivative $D_{\varphi^I}$
\bae{
	D_{\varphi^I}\calO^{JK\cdots}_{LM\cdots}:=&\partial_{\varphi^I}\calO^{JK\cdots}_{LM\cdots}
	+\Gamma^S_{IR}\varpi_S\partial_{\varpi_R}\calO^{JK\cdots}_{LM\cdots} \nonumber \\
	&+\Gamma^J_{IR}\calO^{RK\cdots}_{LM\cdots}+\Gamma^K_{IR}\calO^{JR\cdots}_{LM\cdots}+\cdots \nonumber \\
	&-\Gamma^R_{IL}\calO^{JK\cdots}_{RM\cdots}-\Gamma^R_{IM}\calO^{JK\cdots}_{LR\cdots}-\cdots,
}
and noting that the momentum derivative $\partial_{\pi_I}$ is already covariant,
the \emph{possible} covariant FP equation can be assumed as
\bae{
	\partial_NP=&-D_{\varphi^I}\left(\frac{G^{IJ}}{H}\varpi_JP\right)+\partial_{\varpi_I}\left[\left(3\varpi_I+\frac{V_I}{H}\right)P\right] \nonumber \\
	&+\frac{1}{2}D_{\varphi^I}D_{\varphi^J}(A_{QQ}{}^{IJ}P)+D_{\varphi^I}\partial_{\varpi_J}(A_{Q\tilde{P}}{}^I{}_JP)
	+\frac{1}{2}\partial_{\varpi_I}\partial_{\varpi_J}(A_{\tilde{P}\tilde{P}IJ}P).
}
Here $Q^I$ and $\tilde{P}_I$ represent the UV mode of $\phi^I$ and its covariant momentum ($\tilde{P}_I\approx G_{IJ}D_tQ^J$ in the massless limit)
and $A_{QQ}$, $A_{Q\tilde{P}}$, and $A_{\tilde{P}\tilde{P}}$ indicate their power spectra as noise amplitudes.
This FP equation can be rewritten explicitly as (\Red{TBC})
\bae{
	\partial_NP=&-\partial_{\varphi^I}\left[\left(\frac{\varpi^I}{H}-\frac{1}{2}\Gamma^I_{JK}A_{QQ}{}^{JK}\right)P\right] \nonumber \\
	&-\partial_{\varpi_I}\left[\left(-3\varpi_I-\frac{V_I}{H}+\Gamma^S_{IJ}\frac{\varpi_S\varpi^J}{H}
	-\frac{1}{2}\left(R^S{}_{JIK}-\Gamma^S_{JK,I}\right)\varpi_SA_{QQ}{}^{JK}+\Gamma^J_{IK}A_{Q\tilde{P}}{}^K{}_J\right)P\right] \nonumber \\
	&+\frac{1}{2}\partial_{\varphi^I}\partial_{\varphi^J}(A_{QQ}{}^{IJ}P)+\partial_{\varphi^I}\partial_{\varpi_J}(A_{QP}{}^I{}_JP)
	+\frac{1}{2}\partial_{\varpi_I}\partial_{\varpi_J}(A_{PPIJ}P),
}
where $P_I=\tilde{P}_I+\Gamma_{IJ}^K\varpi_KQ^J$ and $R^S{}_{JIK}$ represents the Riemann tensor
corresponding with the field space metric $G_{IJ}$:
\bae{
    R^S{}_{JIK}=\Gamma^S_{JK,I}-\Gamma^S_{IJ,K}
    +\Gamma^R_{JK}\Gamma^S_{IR}-\Gamma^R_{IJ}\Gamma^S_{KR}.
}

The corresponding It\^o's Langevin equation immediately follows as\footnote{In terms of $A_{QP}$ instead of $A_{Q\tilde{P}}$, $\varpi$'s EoM reads
\bae{
    \dd\varpi_I=\left(-3\varpi_I-\frac{V_I}{H}+\Gamma^S_{IJ}\frac{\varpi_S\varpi^J}{H}+\frac{1}{2}\left(\Gamma^S_{IJ,K}-\Gamma^R_{JK}\Gamma^S_{IR}-\Gamma^R_{IJ}\Gamma^S_{KR}\right)\varpi_SA_{QQ}{}^{JK}+\Gamma^J_{IK}A_{QP}{}^K{}_J\right)\dd N+g_{PIa}\dd W_a.
}}
\bae{
	\bce{
		\dps
		\dd\varphi^I=\left(\frac{\varpi^I}{H}-\frac{1}{2}\Gamma^I_{JK}A_{QQ}{}^{JK}\right)\dd N+g^I_{Qa}\dd W_a, \\[10pt]
		\dps
		\dd\varpi_I=\left(-3\varpi_I-\frac{V_I}{H}+\Gamma^S_{IJ}\frac{\varpi_S\varpi^J}{H}
		-\frac{1}{2}\left(R^S{}_{JIK}-\Gamma^S_{JK,I}\right)\varpi_SA_{QQ}{}^{JK}+\Gamma^J_{IK}A_{Q\tilde{P}}{}^K{}_J\right)\dd N+g_{PIa}\dd W_a.
	}
}
The covariance of this equation is not obvious due to the It\^o's lemma. If one defines the covariant derivatives for $\varphi$ and $\varpi$ in the Graham's way as
\bae{\label{eq: Ito cov derivative}
	\bce{
		\dps
		\calD\varphi^I:=\dd\varphi^I+\frac{1}{2}\Gamma^I_{JK}A_{QQ}{}^{JK}\dd N, \\[10pt]
		\dps
		\calD\varpi_I:=\dd\varpi_I-\Gamma^K_{IJ}\varpi_K\calD\varphi^J
		+\left(\frac{1}{2}\left(R^S{}_{JIK}-\Gamma^S_{JK,I}\right)\varpi_SA_{QQ}{}^{JK}-\Gamma^J_{IK}A_{Q\tilde{P}}{}^K{}_J\right)\dd N,
	}
}
the Langevin equation can be summarized covariantly:
\bae{\label{eq: cov Langevin}
	\bce{
		\dps
		\calD\varphi^I=\frac{\varpi^I}{H}\dd N+g^I_{Qa}\dd W_a, \\[10pt]
		\dps
		\calD\varpi_I=\left(-3\varpi_I-\frac{V_I}{H}\right)\dd N+g_{\tilde{P}Ia}\dd W_a.
	}
}

Let us check that the above Graham's covariant derivatives are in fact covariant against the It\^o's lemma.
Considering the coordinate transformation $\varphi^I\to\tilde{\varphi}^A(\varphi)$, its derivative is given by the It\^o's lemma as
\bae{
	\dd\tilde{\varphi}^A=\pdif{\tilde{\varphi}^A}{\varphi^I}\dd\varphi^I+\frac{1}{2}\frac{\partial^2\tilde{\varphi}^A}{\partial\varphi^I\partial\varphi^J}\dd\varphi^I\dd\varphi^J.
}
On the other hand, the Christoffel symbol transforms as
\bae{\label{eq: christoffel trs.}
	\pdif{\tilde{\varphi}^A}{\varphi^I}\Gamma^I_{JK}=\pdif{\tilde{\varphi}^B}{\varphi^J}\pdif{\tilde{\varphi}^C}{\varphi^K}\Gamma^A_{BC}
	+\frac{\partial^2\tilde{\varphi}^A}{\partial\varphi^J\partial\varphi^K}.
}
Therefore $\calD\varphi^I$ in fact transforms as a vector:
\bae{
	\calD\tilde{\varphi}^A&=\dd\tilde{\varphi}^A+\frac{1}{2}\Gamma^A_{BC}A_{QQ}{}^{BC}\dd N \nonumber \\
	&=\pdif{\tilde{\varphi}^A}{\varphi^I}\dd\varphi^I+\frac{1}{2}\left(\frac{\partial^2\tilde{\varphi}^A}{\partial\varphi^J\partial\varphi^K}
	+\pdif{\tilde{\varphi}^B}{\varphi^J}\pdif{\tilde{\varphi}^C}{\varphi^K}\Gamma^A_{BC}\right)A_{QQ}{}^{JK}\dd N \nonumber \\
	&=\pdif{\tilde{\varphi}^A}{\varphi^I}\calD\varphi^I.
}
The conjugate momentum of this new coordinate is defined by the vectorial transformation as
\bae{
	\pdif{\tilde{\varphi}^A(\varphi)}{\varphi^I}\tilde{\varpi}_A(\varphi,\varpi)=\varpi_I.
}
Noting the arguments in the left side, one derives the following formula by differentiating this equation.
\bae{
	\bce{
		\dps
		\ppdpdpd{\tilde{\varphi}^A}{\varphi^I}{\varphi^J}\tilde{\varpi}_A+\pdif{\tilde{\varphi}^A}{\varphi^I}\pdif{\tilde{\varpi}_A}{\varphi^J}=0, \\[10pt]
		\dps
		\pdif{\tilde{\varphi}^A}{\varphi^I}\pdif{\tilde{\varpi}_A}{\varpi_J}=\delta^J_I, \\[10pt]
		\dps
		\pppdpdpdpd{\tilde{\varphi}^A}{\varphi^I}{\varphi^J}{\varphi^K}\tilde{\varpi}_A+\ppdpdpd{\tilde{\varphi}^A}{\varphi^I}{\varphi^J}\pdif{\tilde{\varpi}_A}{\varphi^K}
		+\ppdpdpd{\tilde{\varphi}^A}{\varphi^I}{\varphi^K}\pdif{\tilde{\varpi}_A}{\varphi^J}+\pdif{\tilde{\varphi}^A}{\varphi^I}\ppdpdpd{\tilde{\varpi}_A}{\varphi^J}{\varphi^K}=0, \\[10pt]
		\dps
		\ppdpdpd{\tilde{\varphi}^A}{\varphi^I}{\varphi^K}\pdif{\tilde{\varpi}_A}{\varpi_J}+\pdif{\tilde{\varphi}^A}{\varphi^I}\ppdpdpd{\tilde{\varpi_A}}{\varpi_J}{\varphi^K}=0, \\[10pt]
		\dps
		\pdif{\tilde{\varphi}^A}{\varphi^I}\ppdpdpd{\tilde{\varpi}_A}{\varpi_J}{\varpi_K}=0.
	}
}
Also from the transformation law of the Christoffel symbol~(\ref{eq: christoffel trs.}), the relation of its derivatives can be obtained as
\bae{
	&\ppdpdpd{\tilde{\varphi}^A}{\varphi^I}{\varphi^J}\Gamma^J_{KL}+\pdif{\tilde{\varphi}^A}{\varphi^J}\Gamma^J_{KL,I} \nonumber \\
	&=\ppdpdpd{\tilde{\varphi}^B}{\varphi^I}{\varphi^K}\pdif{\tilde{\varphi}^C}{\varphi^L}\Gamma^A_{BC}
	+\pdif{\tilde{\varphi}^B}{\varphi^K}\ppdpdpd{\tilde{\varphi}^C}{\varphi^I}{\varphi^L}\Gamma^A_{BC}
	+\pdif{\tilde{\varphi}^B}{\varphi^K}\pdif{\tilde{\varphi}^C}{\varphi^L}\Gamma^A_{BC,I}+\pppdpdpdpd{\tilde{\varphi}^A}{\varphi^I}{\varphi^K}{\varphi^L}.
}
The It\^o's lemma gives the transformation of $\dd\varpi_I$:
\bae{
	\dd\tilde{\varpi}_A=\pdif{\tilde{\varpi}_A}{\varphi^I}\dd\varphi^I+\pdif{\tilde{\varpi}_A}{\varpi_I}\dd\varpi_I
	+\frac{1}{2}\ppdpdpd{\tilde{\varpi}_A}{\varphi^I}{\varphi^J}\dd\varphi^I\dd\varphi^J+\ppdpdpd{\tilde{\varpi}_A}{\varphi^I}{\varpi_J}\dd\varphi^I\dd\varpi_J
	+\frac{1}{2}\ppdpdpd{\tilde{\varpi}_A}{\varpi_I}{\varpi_J}\dd\varpi_I\dd\varpi_J.
}
Combining them and the tensorial transformation of the Riemann tensor $R^S{}_{JIK}$, one finds that the Graham's covariant derivative of the momentum is indeed a vector:
\bae{
	\pdif{\tilde{\varphi}^A}{\varphi^I}\calD\varpi_A=\calD\varpi_I.
}



\section{Path integral derivation}

\subsection{Quadratic action}

Let us try to derive the above correction terms in the first principle path integral approach.
Starting by the general action
\bae{
    S=\int\dd^4x\sqrt{-g}\left[\frac{1}{2}\Mpl^2\calR-\frac{1}{2}g^{\mu\nu}G_{IJ}(\phi)\partial_\mu\phi^I\partial_\nu\phi^J-V(\phi)\right],
}
with the flat-gauged ADM metric
\bae{
    \dd s^2=-N^2\dd t^2+a^2\delta_{ij}(\dd x^i+\beta^i\dd t)(\dd x^j+\beta^j\dd t),
}
we first split the fields into the background IR parts and the perturbed UV modes as
\bae{
    \phi^I=\varphi^I+\delta\phi^I, \qquad \pi_I=\frac{\delta S}{\delta\dot{\phi}^I}=\varpi_I+\delta\pi_I, \qquad N=N_\IR+\alpha, \qquad \beta^i=0+a^{-2}\partial_i\psi.
}
Then the action can be expanded order by order:
\bae{
    S\simeq&\, S^{(0)}+S^{(1)}+S^{(2)}, \\
    S^{(0)}=&\,\int\dd^4x\left[\varpi_I\dot{\varphi}^I-a^3N_\IR\left(\frac{1}{2a^6}\varpi_I\varpi^I+V-\frac{3\Mpl^2\calH^2}{N_\IR^2}\right)\right], \\
    S^{(1)}=&\,\int\dd^4x\left[\delta\tilde{\pi}_I\left(\dot{\varphi}^I-\frac{N_\IR}{a^3}\varpi^I\right)-\delta\phi^I\left(D_t\varpi_I+a^3N_\IR V_I\right)\right. \nonumber \\
    &\left.-a^3\alpha\left(\frac{1}{2a^6}\varpi_I\varpi^I+V-\frac{3\Mpl^2\calH^2}{N_\IR^2}\right)\right], \\
    S^{(2)}=&\int\dd^4x\left[-\frac{3\Mpl^2\calH^2a^3}{N_\IR^3}\alpha^2-a^3\alpha\left(\frac{1}{a^6}\varpi_I\delta\tilde{\pi}^I+V_I\delta\phi^I+\frac{2\Mpl^2\calH}{N_\IR^2}\frac{\nabla^2}{a^2}\psi\right)+\varpi_I\delta\phi^I\frac{\nabla^2}{a^2}\psi\right. \nonumber \\
    &\left.-a^3N_\IR\left(\frac{1}{2a^6}\delta\tilde{\pi}_I\delta\tilde{\pi}^I-\frac{1}{2}\delta\phi_I\frac{\nabla^2}{a^2}\delta\phi^I+\frac{1}{2}V_{I;J}\delta\phi^I\delta\phi^J-\frac{1}{2a^6}R_I{}^{JK}{}_L\varpi_J\varpi_K\delta\phi^I\delta\phi^L\right)+\delta\tilde{\pi}_ID_t\delta\phi^I\right],
}
where
\bae{
    \delta\tilde{\pi}_I=\delta\pi_I-\Gamma^K_{IJ}\varpi_K\delta\phi^J.
}

The important thing is that the field perturbations $\delta\phi^I$ and $\delta\tilde{\pi}_I$ are not covariant but rather expanded as~\cite{Gong:2011uw}
\bae{
    \delta\phi^I=Q^I-\frac{1}{2}\Gamma^I_{JK}Q^JQ^K+\cdots, \qquad \delta\tilde{\pi}_I=\tilde{P}_I+\cdots,
}
with respect to the covariant perturbations $Q^I$ and $\tilde{P}_I$.
Only the linear terms are relevant for $S^{(2)}$, and after substituting the constraint for metric:
\bae{
    \frac{\delta S}{\delta\psi}=0, \quad \Leftrightarrow \quad \alpha=\frac{N_\IR^2}{2\Mpl^2\calH a^3}\varpi_IQ^I,
}
$S^{(2)}$ is summarized as
\bae{
    S^{(2)}&=\frac{1}{2}\int\dd^4x\dd^4x^\prime Q_X^I(x)\Lambda_{XYIJ}(x,x^\prime)Q^J_Y(x^\prime), \\
    Q^I_Q&=Q^I, \qquad Q^I_{\tilde{P}}=\tilde{P}_I, \\
    \bpme{
        \Lambda_{QQIJ}, & \Lambda_{Q\tilde{P}I}{}^J \\
        \Lambda_{\tilde{P}Q}{}^I{}_J, & \Lambda_{\tilde{P}\tilde{P}}{}^{IJ}
    }&=\bpme{
        a^3N_\IR\left(G_{IJ}\frac{\nabla^2}{a^2}-M^2_{QQIJ}\right), & -\delta^J_ID_t-M^2_{\tilde{P}Q}{}^J{}_I \\
        \delta^I_JD_t-M^2_{\tilde{P}Q}{}^I{}_J, & -\frac{N_\IR}{a^3}G^{IJ}
    }\delta^{(4)}(x-x^\prime),
}
where
\bae{
    \bce{
        \dps
        M^2_{QQIJ}=V_{I;J}-\frac{1}{a^6}R_I{}^{KL}{}_J\varpi_K\varpi_L+\frac{N_\IR}{2\Mpl^2\calH a^3}(V_I\varpi_J+V_J\varpi_I)+\frac{3}{2\Mpl^2a^6}\varpi_I\varpi_J, \\
        \dps
        M^2_{\tilde{P}Q}{}^I{}_J=\frac{N_\IR^2}{2\Mpl^2\calH a^6}\varpi^I\varpi_J.
    }
}
The EoM for linear perturbations are given by
\bae{
    \int\dd^4x\Lambda_{XYIJ}Q^I_X=0, \quad \Leftrightarrow \quad \bce{
        \dps
        D_tQ^I=\frac{N_\IR}{a^3}\tilde{P}^I+M^2_{\tilde{P}Q}{}^I{}_JQ^J, \\
        \dps
        D_t\tilde{P}_I=a^3N_\IR\left(G_{IJ}\frac{\nabla^2}{a^2}-M^2_{QQIJ}\right)Q^J-M^2_{\tilde{P}{Q}I}{}^J\tilde{P}_J.
    }
}

For $S^{(1)}$, $\delta\phi^I$ and $\delta\tilde{\pi}_I$ should be expanded up to quadratic order.\footnote{These quadratic terms do not change the EoM for linear perturbations since the corrections are proportional to the classical background EoM which vanishes up to noise.} The relevant terms for IR EFT are only time derivatives as
\bae{
    S^{(1)}\supset\int\dd^4x\left[(\tilde{P}_I+\cdots)\dot{\varphi}^I-\left(Q^I-\frac{1}{2}\Gamma^I_{JK}Q^JQ^K\right)D_t\varpi_I\right].
}
We keep $\delta\tilde{\pi}_I$'s quadratic terms unclear since we haven't yet derived them explicitly.
The IR and UV modes are basically assumed to be decoupled but can be transformed through these couplings.
To clearly see their effects on the IR EoM derived by the variation of the effective action, we simply focus only on the variation of $S^{(1)}$ and interpret these time derivatives as the instant couplings:
\bae{
    \delta S^{(1)}=&\,\int\dd t\int\frac{\dd^3k}{(2\pi^3)}\delta(t-t_\sigma(k))\left[\tilde{P}_I(t,\mathbf{k})\delta\varphi^I(t,-\mathbf{k})+\cdots \right. \nonumber \\
    &\left.\quad-Q^I(t,\mathbf{k})\left(D\varpi_I(t,-\mathbf{k})-\frac{1}{2}\int\dd^3x\ee^{i(\mathbf{k}+\mathbf{k}^\prime)\cdot\mathbf{x}}\Gamma^K_{IJ}(x)D\varpi_K(x)Q^J(t,\mathbf{k}^\prime)\right)\right] \nonumber \\
    =:&\,\int\dd^4x\,Q^I_X(x)\delta\tilde{\varphi}_{XI}(x)+\frac{1}{2}\int\dd^4x\dd^4x^\prime Q_X^I(x)\delta\Lambda_{XYIJ}(x,x^\prime)Q^J_Y(x^\prime),
}
where
\bae{
    D\varpi_I=\delta\varpi_I-\Gamma^K_{IJ}\varpi_K\delta\varphi^J.
}
and in the last equation, we summarize the couplings in a shorthand notation as
\bae{
    \bce{
        \dps
        \delta\tilde{\varphi}_{QI}(x)=\int\frac{\dd^3k}{(2\pi)^3}\ee^{i\mathbf{k}\cdot\mathbf{x}}\delta(t-t_\sigma(k))\delta\varphi_{QI}(t,\mathbf{k})
        :=-\int\frac{\dd^3k}{(2\pi)^3}\ee^{i\mathbf{k}\cdot\mathbf{x}}\delta(t-t_\sigma(k))D\varpi_I(t,\mathbf{k}), \\ 
        \dps
        \delta\tilde{\varphi}_{\tilde{P}I}(x)=\int\frac{\dd^3k}{(2\pi)^3}\ee^{i\mathbf{k}\cdot\mathbf{x}}\delta(t-t_\sigma(k))\delta\varphi_{\tilde{P}I}(t,\mathbf{k})
        :=\int\frac{\dd^3k}{(2\pi)^3}\ee^{i\mathbf{k}\cdot\mathbf{x}}\delta(t-t_\sigma(k))\delta\varphi^I(t,\mathbf{k}), \\
        \dps
        \delta\Lambda_{QQIJ}(x,x^\prime)\supset\Gamma^K_{IJ}(x)D\varpi_K(x)\int\frac{\dd^3k}{(2\pi)^3}\ee^{i\mathbf{k}\cdot(\mathbf{x}-\mathbf{x}^\prime)}\delta(t-t_\sigma(k))\delta(t-t^\prime).
    }
}
$t_\sigma(k)$ represents the crossing time of the coarse-graining scale $k=\sigma aH$.
Accordingly we consider the following action.
\bae{
    S=S^{(0)}+\int\dd^4\,xQ_X^I\delta\tilde{\varphi}_{XI}+\frac{1}{2}\int\dd^4x\dd^4x^\prime Q^I_X(\Lambda_{XYIJ}+\delta\Lambda_{XYIJ})Q^J_Y.
}


\subsection{Schwinger-Keldysh formalism}

As usual, the partition function for the expectation value with respect to the in-vacuum is given by the Schwinger-Keldysh formalism, doubling the degrees of freedom as
\bae{
    &Z[\calJ_X^{Ia},\scrJ_{XIa}]=\calN\int\scrD\delta\varphi_{XIa}\ee^{i(S^{(0)+}-S^{(0)-}+\int\dd^4x\calJ_X^{Ia}\delta\tilde{\varphi}_{XIa})} \nonumber \\
    &\quad\times\int\scrD Q^{Ia}_X\exp\left[i\int\dd^4x(\delta\tilde{\varphi}_{XIa}+\scrJ_{XIa})Q^{Ia}_X+\frac{i}{2}\int\dd^4x\dd^4x^\prime Q^{Ia}_X(\Lambda_{XYIJab}+\delta\Lambda_{XYIJab})Q^{Jb}_Y\right],
}
where the two-dimensional indices $a$ and $b$ take $+$ or $-$ and are raised or lowered by the metric $\sigma_{3ab}=\sigma_3^{ab}=\diag(1,-1)$, and the matrix derivative operators are introduced as
\bae{
    \Lambda_{XYIJab}+\delta\Lambda_{XYIJab}
    =\bpme{
        (\Lambda_{XYIJ}+\delta\Lambda_{XYIJ})(\varphi^+,\varpi^+) & 0 \\ 0 & -(\Lambda_{XYIJ}+\delta\Lambda_{XYIJ})(\varphi^-,\varpi^-)
    }.
}
$\calN$ formally represents the $\delta\varphi$-independent normalization factor.
Integrating $Q$ out, the partition function reads
\bae{
    &Z[\calJ^{Ia}_X,\scrJ_{XIa}]=\frac{\calN}{\sqrt{\Det(1+\Lambda^{-1}\delta\Lambda)}}\int\scrD\delta\varphi_{XIa}\ee^{i(S^{(0)+}-S^{(0)-}+\int\dd^4x\scrJ^{Ia}_X\delta\tilde{\varphi}_{XIa})} \nonumber \\
    &\quad\times\exp\left[-\frac{i}{2}\int\dd^4x\dd^4x^\prime(\delta\tilde{\varphi}_{XIa}+\scrJ_{XIa})(\Lambda+\delta\Lambda)^{-1}{}_{XY}{}^I{}_J{}^a{}_b(\delta\tilde{\varphi}_Y^{Jb}+\scrJ_Y^{Jb})\right].
}
We assume that the $\delta\varphi$ integration simply replaces $\delta\varphi$ by its vacuume expectation value $\braket{\delta\varphi}$ as IR modes are expected to be classicalized well.
Therefore we obtain
\bae{
    Z[\calJ^{Ia}_X,\scrJ_{XIa}]\simeq&\,\frac{\calN}{\sqrt{\Det(1+\Lambda^{-1}\delta\Lambda)}}\exp\left[i\left(S^{(0)+}-S^{(0)-}+\int\dd^4x\calJ_X^{Ia}\braket{\delta\tilde{\varphi}_{XIa}}\right)\right. \nonumber \\
    &\left.-\frac{1}{2}\int\dd^4x\dd^4x^\prime(\braket{\delta\tilde{\varphi}_{XIa}}+\scrJ_{XIa})(\Lambda+\delta\Lambda)^{-1}{}_{XY}{}^I{}_J{}^a{}_b(\braket{\delta\tilde{\varphi}_Y^{Jb}}+\scrJ_Y^{Jb})\right] \nonumber \\
    =:&\,\calN\ee^{iW[\calJ_X^{Ia},\scrJ_{XIa}]}.
}
The effective action $\Gamma$ is given by the Legendre transformation of $W$ as
\bae{
    \Gamma[\braket{\delta\varphi_{XIa}},\braket{Q_X^{Ia}}]=W[\calJ_X^{Ia},\scrJ_{XIa}]-\int\dd^4x\calJ_X^{Ia}\braket{\delta\tilde{\varphi}_{XIa}}-\int\dd^4x\scrJ_{XIa}\braket{Q_X^{Ia}},
}
where
\bae{
    \braket{Q_X^{Ia}}=\frac{\delta W}{\delta\scrJ_{XIa}}=-\int\dd^4x^\prime(\Lambda+\delta\Lambda)^{-1}{}_{XY}{}^I{}_J{}^a{}_b(\braket{\delta\tilde{\varphi}_Y^{Jb}}+\scrJ_Y^{Jb}).
}
Therefore the explicit form of $\Gamma$ reads
\bae{
    \Gamma[\braket{\delta\varphi_{XIa}},\braket{Q_X^{Ia}}]=&\,S^{(0)+}-S^{(0)-}+\frac{i}{2}\log\Det(1+\Lambda^{-1}\delta\Lambda) \nonumber \\
    &+\frac{1}{2}\int\dd^4x\dd^4x^\prime\braket{Q_X^{Ia}}(\Lambda+\delta\Lambda)_{XYIJab}\braket{Q_Y^{Jb}}+\int\dd^4x\braket{\delta\tilde{\varphi}_{XIa}}\braket{Q_X^{Ia}}.
}
The vacuum expectation value $\braket{Q}$ is given by the stationary point of the effective action as
\bae{
    \frac{\delta\Gamma}{\delta\braket{Q_X^{Ia}}}=0, \quad \Leftrightarrow \quad \braket{Q_X^{Ia}}=-\int\dd^4x^\prime(\Lambda+\delta\Lambda)^{-1}{}_{XY}{}^I{}_J{}^a{}_b\braket{\delta\tilde{\varphi}_X^{Jb}}
}
Substituting this solution back into the effective action and making use of the formula $\log\Det A=\Tr\log A$ and the linear expansion $\log(1+A)\simeq A$, one finally obtains the IR effective action, at the leading order in $\delta\varphi$, as
\bae{\label{eq: GammaIR}
    \Gamma_\IR&\simeq S^{(0)+}-S^{(0)-}+\frac{i}{2}\Tr(\Lambda^{-1}\delta\Lambda)-\frac{1}{2}\int\dd^4x\dd^4x^\prime\braket{\delta\tilde{\varphi}_{XIa}}\Lambda^{-1}{}_{XY}{}^I{}_J{}^a{}_b\braket{\delta\tilde{\varphi}_Y^{Jb}} \nonumber \\
    &=:S^{(0)+}-S^{(0)-}+\delta S_\IA^\mathrm{det}+\delta S_\IA^\xi
}
The UV-correction $\delta S_\IA^\mathrm{det}$ and $\delta S_\IA^\xi$ are often called \emph{influence action}.
Hereafter we will omit brackets.




\subsection{Influence action}

Let us first discuss the second influence action $\delta S_\IA^\xi$.
$\Lambda^{-1}$ is nothing but the closed-time-path ordered two point function of $Q$ as
\bae{
	\braket{T_C\hat{Q}_X^{Ia}(x)\hat{Q}_Y^{Jb}(x^\prime)}=\left.\frac{1}{i}\frac{\delta}{\delta\scrJ_{XIa}(x)}\frac{1}{i}\frac{\delta}{\delta\scrJ_{YJb}(x^\prime)}Z\right|_{\calJ=\scrJ=0}=i\Lambda^{-1}{}_{XY}{}^{IJab}(x,x^\prime),
}
where the time ordering is defined by
\bae{
	T_C\hat{Q}_{X}^{Ia}(x)\hat{Q}_Y^{Jb}(x^\prime)=
	\bce{
		\dps
		\theta(t-t^\prime)\hat{Q}_{X}^{Ia}(x)\hat{Q}_Y^{Jb}(x^\prime)+\theta(t^\prime-t)\hat{Q}_Y^{Jb}(x^\prime)\hat{Q}_X^{Ia}(x), & (a=+,\,b=+), \\
		\dps
		\hat{Q}_Y^{Jb}(x^\prime)\hat{Q}_X^{Ia}(x), & (a=+,\,b=-), \\
		\dps
		\hat{Q}_X^{Ia}(x)\hat{Q}_Y^{Jb}(x^\prime), & (a=-,\,b=+), \\
		\dps
		\theta(t^\prime-t)\hat{Q}_{X}^{Ia}(x)\hat{Q}_Y^{Jb}(x^\prime)+\theta(t-t^\prime)\hat{Q}_Y^{Jb}(x^\prime)\hat{Q}_X^{Ia}(x), & (a=-,\,b=-).
	}
}
Therefore, expanding the field operators by the mode functions as
\bae{\label{eq: mode expansion}
	\hat{Q}_X^{Ia}(x)=\int\dk\left[Q_{X\alpha}^{Ia}(t,k)\hat{a}_\alpha(\mathbf{k})\ee^{i\mathbf{k}\cdot\mathbf{x}}
	+Q_{X\alpha}^{*Ia}(t,k)\hat{a}^\dagger_\alpha(\mathbf{k})\ee^{-i\mathbf{k}\cdot\mathbf{x}}\right],
}
with the annihilation-creation operators
\bae{\label{eq: aadagger}
	[\hat{a}_\alpha(\mathbf{k}),\hat{a}^\dagger_\beta(\mathbf{q})]=(2\pi)^3\delta_{\alpha\beta}\delta^{(3)}(\mathbf{k}-\mathbf{q}),
}
the explicit form of $\Lambda^{-1}$ is given by
\bae{
	&\Lambda^{-1}{}_{XY}{}^{IJab}(x,x^\prime)=-i\braket{T_C\hat{Q}_X^{Ia}(x)\hat{Q}_Y^{Jb}(x^\prime)} \nonumber \\
	&=-i\int\dk\bpme{
		\begin{array}{c}
			\theta(t-t^\prime)Q_{X\alpha}^I(t,k)Q^{*J}_{Y\alpha}(t^\prime,k)\ee^{i\mathbf{k}\cdot(\mathbf{x}-\mathbf{x}^\prime)} \\
			+\theta(t^\prime-t)Q_{Y\alpha}^J(t^\prime,k)Q_{X\alpha}^{*I}(t,k)\ee^{i\mathbf{k}\cdot(\mathbf{x}^\prime-\mathbf{x})}
		\end{array}
		& Q^J_{Y\alpha}(t^\prime,k)Q_{X\alpha}^{*I}(t,k)\ee^{i\mathbf{k}\cdot(\mathbf{x}^\prime-\mathbf{x})} \\
		Q_{X\alpha}^I(t,k)Q^{*J}_{Y\alpha}(t^\prime,k)\ee^{i\mathbf{k}\cdot(\mathbf{x}-\mathbf{x}^\prime)} &
		\begin{array}{c}
			\theta(t^\prime-t)Q_{X\alpha}^I(t,k)Q^{*J}_{Y\alpha}(t^\prime,k)\ee^{i\mathbf{k}\cdot(\mathbf{x}-\mathbf{x}^\prime)} \\
			+\theta(t-t^\prime)Q^J_{Y\alpha}(t^\prime,k)Q_{X\alpha}^{*I}(t,k)\ee^{i\mathbf{k}\cdot(\mathbf{x}^\prime-\mathbf{x})}
		\end{array}
	}.
}
Note that we used the same mode function for $+/-$ modes because we later take the $\varphi^+=\varphi^-$ limit to obtain the IR EoM. 
Here the so-called Keldysh basis is useful rather than $+/-$ one:
\bae{
	\bpme{
		\delta\bar{\varphi}_X^I \\
		\delta\varphi_X^{I\Delta}
	} = \bpme{
		1/2 & 1/2 \\
		1 & -1
	}\bpme{
		\delta\varphi_X^{I+} \\
		\delta\varphi_X^{I-}
	}.
}
In this basis, the propagator reads
\bae{
    &\Lambda^{-1}_{XY}{}^{IJ}(x,x^\prime) \nonumber \\
    &=-i\int\frac{\dd^3k}{(2\pi^3)}\bpme{
        0 & 2i\theta(t^\prime-t)\Im[Q_{Y\alpha}^J(t^\prime,k)Q_{X\alpha}^{*I}(t,k)\ee^{i\mathbf{k}\cdot(\mathbf{x}^\prime-\mathbf{x})}] \\
        2i\theta(t-t^\prime)\Im[Q_{X\alpha}^I(t,k)Q_{Y\alpha}^{*J}(t^\prime,k)\ee^{i\mathbf{k}\cdot(\mathbf{x}-\mathbf{x}^\prime)}] & \Re[Q_{X\alpha}^I(t,k)Q_{Y\alpha}^{*J}(t^\prime,k)\ee^{i\mathbf{k}\cdot(\mathbf{x}-\mathbf{x}^\prime)}]
    }
}
Then $\delta S_\IA^\xi$ is summarized as
\bae{
	\delta S_\IA^\xi=&\,\frac{i}{2}\int\dd^4x\dd^4x^\prime\delta\varphi_{XI}^\Delta(x)\Re[\Pi_{XY}{}^{IJ}(x,x^\prime)]\delta\varphi_{YJ}^\Delta(x^\prime) \nonumber \\
	&-2\int\dd^4x\dd^4x^\prime\theta(t-t^\prime)\delta\varphi_{XI}^\Delta(x)\Im[\Pi_{XY}{}^{IJ}(x,x^\prime)]\delta\bar{\varphi}_{YJ}(x^\prime),
}
where
\bae{
	\Pi_{XY}{}^{IJ}(x,x^\prime)&=\int\dk\ee^{i\mathbf{k}\cdot(\mathbf{x}-\mathbf{x}^\prime)}\delta(t-t_\sigma)\delta(t^\prime-t_\sigma)
	Q_{X\alpha}^I(t,k)Q^{*J}_{Y\alpha}(t^\prime,k) \nonumber \\
	&=\frac{\dot{k}_\sigma}{k_\sigma}\frac{\sin k_\sigma r}{k_\sigma r}\calP_{XY}{}^{IJ}(k_\sigma)\delta(t-t^\prime),
}
with
\bae{\label{eq: calP}
	\calP_{XY}{}^{IJ}(k)=\frac{k^3}{2\pi^2}Q_{X\alpha}^I(k)Q_{Y\alpha}^{*J}(k).
}
The second term is simply quadratic in $\delta\varphi$ and neglected in the IR EoM. However the first term is imaginary and therefore rather interpreted as a Gaussian integral
by introducing the auxiliary fields $\xi$:
\bae{\label{eq: functional Fourier trs}
	\ee^{iS_\IA^\xi}=\calN\int\scrD\xi_X^IP[\xi_X^I]\ee^{i\int\dd^4x\xi_X^I\delta\varphi_{XI}^\Delta},
}
where
\bae{
	P[\xi_X^I]=\calN\exp\left[-\frac{1}{2}\int\dd^4x\dd^4x^\prime\xi_{X}^I(x)(\Re\Pi^{-1})_{XYIJ}(x,x^\prime)\xi_Y^J(x^\prime)\right].
}
The Gaussian weight $P[\xi_X^I]$ determines the statistical properties of $\xi$ as
\bae{
	\bce{
		\dps
		\braket{\xi_X^I(x)}=\int\scrD\xi_X^I\,\xi_X^I(x)P[\xi_X^I]=0, \\[10pt]
		\dps
		\braket{\xi_X^I(x)\xi_{Y}^J(x^\prime)}=\int\scrD\xi_X^I\,\xi_X^I(x)\xi_{Y}^J(x^\prime)P[\xi_X^I]=\Re\Pi_{XY}{}^{IJ}(x,x^\prime),
	}
}
and the other part in Eq.~(\ref{eq: functional Fourier trs}) is interpreted as the couplings between $\xi$ and $\varphi$.
Then this part is rather considered as a true IR effective action
\bae{
	\delta S_\eff^\xi=\int\dd^4x\xi_X^I\delta\varphi_{XI}^\Delta=\int\dd^4x\left[\xi_Q^I(-\delta\varpi_I^\Delta+\Gamma^K_{IJ}\varpi_K\delta\varphi^{J\Delta})
	+\xi_{\tilde{P}I}\delta\varphi^{I\Delta}\right].
}

\bigskip

Let us also consider $\delta S_\IA^\mathrm{det}$.
Again taking the $\varphi^+=\varphi^-$ limit except for $\delta\varphi$, the matrix elements of $\delta\Lambda$ is given by
\bae{
    \delta\Lambda_{QQIJab}(x,x^\prime)\supset&\,\Gamma^K_{IJ}(x)\int\frac{\dd^3k}{(2\pi)^3}\ee^{i\mathbf{k}\cdot(\mathbf{x}-\mathbf{x}^\prime)}\delta(t-t_\sigma(k))\delta(t-t^\prime) \nonumber \\
    &\times\bpme{
        D\bar{\varpi}_K(x)+\frac{1}{2}D\varpi_K^\Delta(x) & 0 \\
        0 & -D\bar{\varpi}_K(x)+\frac{1}{2}D\varpi_K^\Delta(x)
    }.
}
Therefore $\delta S_\IA^\mathrm{det}$ can be summarized as
\bae{
    \delta S_\IA^\mathrm{det}=\frac{i}{2}\Tr(\Lambda^{-1}\delta\Lambda)\supset\frac{1}{2}\Re\Pi_{QQ}{}^{IJ}(x,x)\int\dd^4x\Gamma_{IJ}^KD\varpi_K^\Delta.
}
Therefore the covariant EoM for $\varphi$ reads
\bae{
	&\left.\frac{\delta S^{(0)+}-\delta S^{(0)-}+\delta S_\IA^\mathrm{det}+\delta S_\eff^\xi}{\delta\varpi_I^\Delta}\right|_{\varphi^\Delta=\varpi^\Delta=0}=0, \nonumber \\ &\qquad \Leftrightarrow
	\quad \frac{\calD\varphi^I}{\dd t}:=\dot{\varphi}^I+\frac{1}{2}\Gamma^I_{JK}\Re\Pi_{QQ}{}^{JK}=\frac{N_\IR}{a^3}\varpi^I+\xi_Q^I.
}
On the other hand $\varpi$'s EoM is given, at this stage, by
\bae{
    &\left.\frac{\delta S^{(0)+}-\delta S^{(0)-}+\delta S_\IA^\mathrm{det}+\delta S_\eff^\xi}{\delta\varphi^{I\Delta}}\right|_{\varphi^\Delta=\varpi^\Delta=0}=0, \nonumber \\
    &\qquad \Leftrightarrow \quad \dot{\varpi}_I-\Gamma_{IJ}^K\varpi_K\xi_Q^J+\frac{1}{2}\Gamma^S_{IR}\Gamma^R_{JK}\varpi_S\Re\Pi_{QQ}{}^{JK}=-a^3N_\IR V_I+\frac{N_\IR}{a^3}\Gamma_{IJ}^K\varpi^J\varpi_K+\xi_{\tilde{P}I}, \nonumber \\
    &\qquad \Leftrightarrow \quad
    \dot{\varpi}_I-\Gamma^K_{IJ}\varpi_K\frac{\calD\varphi^I}{\dd t}+\frac{1}{2}\Gamma^S_{IR}\Gamma^R_{JK}\varpi_S\Re\Pi_{QQ}{}^{JK}=-a^3N_\IR V_I+\xi_{\tilde{P}I}.
}
Comparing this equation with Eqs.~(\ref{eq: Ito cov derivative}) and (\ref{eq: cov Langevin}),
one finds that the following terms are needed for $\delta\Lambda$.
\bae{
    \delta\Lambda_{QQIJ}&\supset(\Gamma^S_{KI,J}+\Gamma^R_{KI}\Gamma^S_{JR})\varpi_S\delta\varphi^K\int\frac{\dd^3k}{(2\pi)^3}\ee^{i\mathbf{k}\cdot(\mathbf{x}-\mathbf{x}^\prime)}\delta(t-t_\sigma(k))\delta(t-t^\prime), \\
    \delta\Lambda_{Q\tilde{P}I}{}^J&=\Gamma^J_{IK}\delta\varphi^K\int\frac{\dd^3k}{(2\pi)^3}\ee^{i\mathbf{k}\cdot(\mathbf{x}-\mathbf{x}^\prime)}\delta(t-t_\sigma(k))\delta(t-t^\prime).
}
Therefore the field perturbation $\delta\tilde{\pi}_I$ should be expanded up to quadratic order as
\bae{\label{eq: expansion of delta pi}
    \delta\tilde{\pi}_I=\tilde{P}_I+\frac{1}{2}(\Gamma^S_{IJ,K}+\Gamma^R_{IJ}\Gamma^S_{KR})\varpi_SQ^JQ^K+\Gamma^K_{IJ}Q^J\tilde{P}_K,
}
and $S^{(1)}$'s time derivatives are given by
\bae{
    S^{(1)}\supset&\int\dd^4x\left[\left(\tilde{P}_I+\frac{1}{2}(\Gamma^S_{IJ,K}+\Gamma^R_{IJ}\Gamma^S_{KR})\varpi_SQ^JQ^K+\Gamma^K_{IJ}Q^J\tilde{P}_K\right)\dot{\varphi}^I \right. \nonumber \\
    &\quad\left.-\left(Q^I-\frac{1}{2}\Gamma^I_{JK}Q^JQ^K\right)D_t\varpi_I\right]
}


\subsection{Quadratic expansion of momentum perturbation}

Let us derive the quadratic expansion of $\delta\tilde{\pi}_I$~(\ref{eq: expansion of delta pi}), following Ref.~\cite{Gong:2011uw}.
For the field perturbation $\delta\phi^I$, they first introduced the affine parameter $\lambda$ and then the field displacement $\phi^I(\lambda=\epsilon)-\phi^I(\lambda=0)$ is related to the covariant perturbation $Q^I$ so that $\phi^I(\lambda)$ represents the geodesic ``free fall" trajectory:
\bae{\label{eq: free fall for phi}
    D_\lambda^2\phi^I=\dif{{}^2\phi^I}{\lambda^2}+\Gamma^I_{JK}\dif{\phi^J}{\lambda}\dif{\phi^K}{\lambda}=0,
}
with the initial ``velocity"
\bae{
    \left.\dif{\phi^I}{\lambda}\right|_{\lambda=0}=Q^I.
}
That is, the displacement can be expanded as
\bae{
    \phi^I(\lambda=\epsilon)-\phi^I(\lambda=0)=\left.\dif{\phi^I}{\lambda}\right|_{\lambda=0}\epsilon+\frac{1}{2}\left.\dif{{}^2\phi^I}{\lambda^2}\right|_{\lambda=0}\epsilon^2=Q^I\epsilon-\frac{1}{2}\Gamma^I_{JK}Q^JQ^K\epsilon^2,
}
up to quadratic order. Setting $\epsilon=1$, one obtains the quadratic expansion of the field perturbation
\bae{
    \delta\phi^I:=\phi^I(\lambda=1)-\phi^I(\lambda=0)=Q^I-\frac{1}{2}\Gamma^I_{JK}Q^JQ^K.
}

Also for the momentum perturbation, one follows the same procedure. The momentum displacement is defined by the ``free fall" trajectory (as well as Eq~(\ref{eq: free fall for phi}))
\bae{
    0=D_\lambda^2\pi_I&=D_\lambda\left(\dif{\pi_I}{\lambda}-\Gamma^K_{IJ}\pi_K\dif{\phi^J}{\lambda}\right) \nonumber \\
    &=\dif{}{\lambda}\left(\dif{\pi_I}{\lambda}-\Gamma^K_{IJ}\pi_K\dif{\phi^J}{\lambda}\right)-\Gamma^K_{IJ}\dif{\phi^J}{\lambda}\left(\dif{\pi_K}{\lambda}-\Gamma^S_{KL}\pi_S\dif{\phi^L}{\lambda}\right) \nonumber \\
    &=\dif{{}^2\pi_I}{\lambda^2}-2\Gamma^K_{IJ}\dif{\phi^J}{\lambda}\dif{\pi_K}{\lambda}-(\Gamma^S_{IJ,K}-\Gamma^S_{IR}\Gamma^R_{JK}-\Gamma^R_{IJ}\Gamma^S_{RK})\pi_S\dif{\phi^J}{\lambda}\dif{\phi^K}{\lambda},
}
with the initial ``velocity"
\bae{
    \left.\dif{\phi^I}{\lambda}\right|_{\lambda=0}=Q^I, \qquad \left.\dif{\pi_I}{\lambda}\right|_{\lambda=0}=P_I.
}
Then, with use of the Taylor expansion up to quadratic order, the momentum perturbation is given by
\bae{
    \delta\pi_I&:=\pi_I(\lambda=1)-\pi_I(\lambda=0)=\left.\dif{\pi_I}{\lambda}\right|_{\lambda=0}+\frac{1}{2}\left.\dif{{}^2\pi_I}{\lambda^2}\right|_{\lambda=0} \nonumber \\
    &=P_I+\Gamma^K_{IJ}Q^JP_K+\frac{1}{2}(\Gamma^S_{IJ,K}-\Gamma^S_{IR}\Gamma^R_{JK}-\Gamma^R_{IJ}\Gamma^S_{RK})\pi_SQ^JQ^K.
}
Substituting $P_I=\tilde{P}_I+\Gamma^K_{IJ}\pi_KQ^J$, one obtains Eq.~(\ref{eq: expansion of delta pi})
\bae{
    \delta\tilde{\pi}_I=\delta\pi_I-\Gamma^K_{IJ}\pi_K\mathblue{\delta\phi^J}=\tilde{P}_I+\frac{1}{2}(\Gamma^S_{IJ,K}+\Gamma^R_{IJ}\Gamma^S_{RK})\pi_SQ^JQ^K+\Gamma^K_{IJ}Q^J\tilde{P}_K.
}
Note that the background value $\pi_I(\lambda=0)$ should be interpreted as $\varpi_I$ in the IR-UV decomposition.



\section{Cubic action}

There seems an ambiguity in the correspondence between $(\dphi,\dpi)$ and $(Q,P)$ beyond the linear order.
This ambiguity does not affect the linear perturbation EoM and thus the quadratic action, but does change the cubic action.
In this section, we derive the cubic action with our ``free fall" extrapolation $D_\lambda^2\phi^I=0=D_\lambda^2\pi_I$ and investigate its relation to the ordinary cubic Hamiltonian action (e.g. given by Ref.~\cite{Butchers:2018hds}).

With use of the ADM decomposition,
\bae{
    \dd s^2=-\calN^2\dd t^2+\gamma_{ij}(\dd x^i+\beta^i\dd t)(\dd x^j+\beta^j\dd t),
}
the Hamiltonian action is given by
\bae{
    S=\int\dd^4x(\pi^{ij}\dot{\gamma}_{ij}+\pi_I\dot{\phi}^I-\scrH),
}
with the Hamiltonian density
\bae{
    \scrH=\calN(C^\phi+C^G)+\beta^i(C_i^\phi+C_i^G),
}
where
\bae{
    \begin{array}{ll}
        \dps
        C^\phi=\sqrt{\gamma}\left[\frac{G^{IJ}}{2\gamma}\pi_I\pi_J+\frac{G_{IJ}}{2}\gamma^{ij}\partial_i\phi^I\partial_j\phi^J+V\right], & 
        \dps\qquad
        C^G=\frac{2}{\sqrt{\gamma}\Mpl^2}\left[\pi_{ij}\pi^{ij}-\frac{1}{2}\left(\pi^i_i\right)^2-\frac{\gamma}{4}\Mpl^4\calR^{(3)}\right], \\
        C^\phi_i=\pi_I\partial_i\phi^I, &
        \dps\qquad
        C^G_i=-2\partial_k(\gamma_{ij}\pi^{jk})+\pi^{jk}\partial_i\gamma_{jk}.
    \end{array}
}
$\calR^{(3)}$ is the curvature of the spatial hypersurfaces.
If one neglects the tensor perturbations, there is no dynamical mode in the metric part and therefore one can substitute back the on-shell condition for $\pi^{ij}$,
\bae{
    \pi^{ij}=\var{S}{\dot{\gamma}_{ij}}=\frac{\Mpl^2}{2}\sqrt{\gamma}(K\gamma^{ij}-K^{ij}),
}
where
\bae{
    K_{ij}=\frac{1}{2\calN}(\beta_{i,j}+\beta_{j,i}-\dot{\gamma}_{ij}), \qquad K=K^i_i.
}
The spatial indices $i,j,\cdots$ are raised and lowered by the spatial metric $\gamma_{ij}$ and its inverse $\gamma^{ij}$.
Taking the spatially flat gauge $\gamma_{ij}=a^2(t)\delta_{ij}$ (i.e. $\gamma^{ij}=a^{-2}(t)\delta_{ij}$, $\gamma=a^6(t)$, and $\calR^{(3)}=0$), we show more explicit forms of them.
\bae{
    K_{ij}&=\frac{1}{2\calN}(\beta_{i,j}+\beta_{j,i}-2a^2\calH\delta_{ij}), \qquad K=\frac{1}{a^2\calN}(\beta_{i,i}-3a^2\calH), \\
    \pi^{ij}&=\frac{\Mpl^2}{2a\calN}\left[\beta_{k,k}\delta_{ij}-\frac{1}{2}(\beta_{i,j}+\beta_{j,i})-2a^2\calH\delta_{ij}\right], \\
    \pi^{ij}\dot{\gamma}_{ij}&=\frac{2a\Mpl^2\calH}{\calN}(\beta_{i,i}-3a^2\calH).
}
$\calH=\dot{a}/a$ is a generalized Hubble parameter for an arbitrary choice of the time variable.
Hamiltonian constraints also read
\bae{
    C^\phi&=a^3\left[\frac{G^{IJ}}{2a^6}\pi_I\pi_J+\frac{G_{IJ}}{2a^2}\partial_i\phi^I\partial_i\phi^J+V\right], \\
    C^G&=-\frac{\Mpl^2}{2a\calN^2}\left[(\beta_{i,i})^2-\frac{1}{2}(\beta_{i,j}\beta_{i,j}+\beta_{i,j}\beta_{j,i})-4a^2\calH\beta_{i,i}+6a^4\calH^2\right], \\
    C_i^\phi&=\pi_I\partial_i\phi^I, \\
    C_i^G&=-\frac{a\Mpl^2}{\calN^2}\left[\calN\left(\beta_{j,ji}-\frac{1}{2}(\beta_{i,jj}+\beta_{j,ij})\right)-\calN_i(\beta_{j,j}-2a^2\calH)+\frac{1}{2}\calN_j(\beta_{i,j}+\beta_{j,i})\right].
}
Note that $\pi^{ij}\dot{\gamma}_{ij}$ and $\calN C^G$ can be simply summarized as
\bae{
	\pi^{ij}\dot{\gamma}_{ij}-\calN C^G=\frac{\Mpl^2}{2a\calN}\left[(\beta_{i,i})^2-\frac{1}{2}(\beta_{i,j}\beta_{i,j}+\beta_{i,j}\beta_{j,i})-6a^4\calH^2\right].
}

We then decompose the physical variables into the IR and UV modes:
\bae{
    \phi^I=\varphi^I+\dphi^I, \qquad \pi_I=\varpi_I+\dpi_I, \qquad \calN=\calN_\IR+\alpha, \qquad \beta^i=0+a^{-2}\partial_i\psi.
}
The order by order expansion of each quantity in UV modes can be summarized as
\bae{
	&\bce{
    		\dps
 	   	(\pi^{ij}\dot{\gamma}_{ij}-\calN C^G)^{(0)} &
		\dps
		=-\frac{3a^3\Mpl^2\calH^2}{\calN_\IR}, \\
		\dps
		(\pi^{ij}\dot{\gamma}_{ij}-\calN C^G)^{(1)} &
		\dps
		=\frac{3a^3\Mpl^2\calH^2}{\calN_\IR^2}\alpha, \\
		\dps
		(\pi^{ij}\dot{\gamma}_{ij}-\calN C^G)^{(2)} & 
		\dps
		=-\frac{3a^3\Mpl^2\calH^2}{\calN_\IR^3}\alpha^2+\frac{\Mpl^2}{2a\calN_\IR}[(\nabla^2\psi)^2-\psi_{ij}\psi_{ij}], \\
		\dps
		(\pi^{ij}\dot{\gamma}_{ij}-\calN C^G)^{(3)} & 
		\dps
		=\frac{3a^3\Mpl^2\calH^2}{\calN_\IR^4}\alpha^3-\frac{\Mpl^2}{2a\calN_\IR^2}\alpha[(\nabla^2\psi)^2-\psi_{ij}\psi_{ij}],
	    } \\
    &\bce{
        \dps
        (\pi_I\dot{\phi}^I)^{(0)} & 
        \dps
        =\varpi_I\dot{\varphi}^I, \\
        \dps
        (\pi_I\dot{\phi}^I)^{(1)} &
        \dps
        =\dot{\varphi}^I\dpi_I+\varpi_I\delta\dot{\phi}^I=\dot{\varphi}^I\Dpi_I+\varpi_ID_t\dphi^I, \\
        \dps
        (\pi_I\dot{\phi}^I)^{(2)} &
        \dps
        =\dpi_I\delta\dot{\phi}^I \\
        &\dps
        =\Dpi_ID_t\dphi^I+\Gamma^K_{IJ}\varpi_K\dphi^ID_t\dphi^J-\Gamma^I_{JK}\dot{\varphi}^J\dphi^K\Dpi_I-\Gamma^I_{JK}\Gamma^L_{IM}\dot{\varphi}^J\varpi_L\dphi^K\dphi^M, \\
        &\dps
        \to\Dpi_ID_t\dphi^I-\frac{1}{2}(\Gamma^K_{IJ,L}+\Gamma^K_{LS}\Gamma^S_{IJ})\dot{\varphi}^L\varpi_K\dphi^I\dphi^J \\
        &\dps \qquad
        -\frac{1}{2}\Gamma^K_{IJ}(D_t\varpi_K)\dphi^I\dphi^J-\Gamma^I_{JK}\dot{\varphi}^J\dphi^K\Dpi_I, \\
        \dps
        (\pi_I\dot{\phi}^I)^{(3)} &
        \dps
        =0,
    } \\
    &\bce{
        \dps
        {C^\phi}^{(0)} &
        \dps
        =a^3\left[\frac{1}{2a^6}\varpi_I\varpi^I+V\right], \\
        \dps
        {C^\phi}^{(1)} &
        \dps
        =a^3\left[\frac{G^{IJ}{}_{,K}}{2a^6}\varpi_I\varpi_J\dphi^K+\frac{G^{IJ}}{a^6}\varpi_I\dpi_J+V_I\dphi^I\right]=a^3\left[\frac{1}{a^6}\varpi^I\Dpi_I+V_I\dphi^I\right], \\
        \dps
        {C^\phi}^{(2)} &
        \dps
        =a^3\left[\frac{G^{IJ}{}_{,KL}}{4a^6}\varpi_I\varpi_J\dphi^K\dphi^L+\frac{G^{IJ}{}_{,K}}{a^6}\varpi_I\dpi_J\dphi^K+\frac{G^{IJ}}{2a^6}\dpi_I\dpi_J \right. \\
        & \dps
        \left. \qquad +\frac{G^{IJ}}{2a^2}\partial_i\dphi^I\partial_i\dphi^J+\frac{1}{2}V_{IJ}\dphi^I\dphi^J\right] \\
        &\dps
        =a^3\left[\frac{1}{2a^6}\Dpi_I\Dpi^I+\frac{1}{2a^2}\partial_i\dphi^I\partial_i\dphi_I+\frac{1}{2}V_{I;J}\dphi^I\dphi^J
        -\frac{1}{2a^6}R_I{}^{JK}{}_L\varpi_J\varpi_K\dphi^I\dphi^L \right. \nonumber \\
        &\dps\qquad
        \left.-\frac{1}{a^6}\Gamma^K_{IJ}\varpi^I\dphi^J\Dpi_K-\frac{1}{2a^6}(\Gamma^I_{KL,J}+\Gamma^I_{JS}\Gamma^S_{KL})\varpi_I\varpi^J\dphi^K\dphi^L
        +\frac{1}{2}\Gamma^K_{IJ}V_K\dphi^I\dphi^J\right], \\
        \dps
        {C^\phi}^{(3)} &
        \dps
        =a^3\left[\frac{G^{IJ}{}_{,KLM}}{12a^6}\varpi_I\varpi_J\dphi^K\dphi^L\dphi^M+\frac{G^{IJ}{}_{,KL}}{2a^6}\varpi_I\dpi_J\dphi^K\dphi^L
        +\frac{G^{IJ}{}_{,K}}{2a^6}\dpi_I\dpi_J\dphi^K \right. \\
        &\dps
        \left.\qquad+\frac{G_{IJ,K}}{2a^2}\dphi^K\partial_i\dphi^I\partial_i\dphi^J+\frac{V_{IJK}}{6}\dphi^I\dphi^J\dphi^K\right] \\
        &\dps
        =a^3\left[\frac{1}{a^6}\Gamma^I_{JK}\dphi^K\partial_i\dphi_I\partial_i\dphi^K-\frac{1}{a^6}\Gamma^I_{JK}\Dpi_I\Dpi^J\dphi^K \right. \\
        &\dps\qquad
        -\frac{1}{a^6}\left(R_K{}^{IJ}{}_L+\Gamma^I_{KL,R}G^{JR}-\Gamma^I_{LS}\Gamma^S_{KR}G^{JR}+\Gamma^I_{RS}\Gamma^S_{KL}G^{JR}\right)\varpi_I\Dpi_J\dphi^K\dphi^L \\
        &\dps\qquad-\frac{1}{6a^6}\left(R_K{}^{IJ}{}_{L;M}+2\Gamma^S_{KL}R_S{}^{IJ}{}_M+4\Gamma^I_{SK}R_L{}^{JS}{}_M+\Gamma^I_{KL,MS}G^{JS} \right.\\
        &\dps\qquad\qquad+\Gamma^I_{KL,S}(5\Gamma^J_{MR}G^{RS}-\Gamma^S_{MR}G^{JR})
        -2(\Gamma^I_{KS,M}\Gamma^S_{LR}+\Gamma^I_{KS}\Gamma^S_{LR,M})G^{JR} \\
        &\dps\qquad\qquad
        +(\Gamma^I_{RS,K}\Gamma^S_{LM}+\Gamma^I_{RS}\Gamma^S_{LM,K})G^{JR}-\Gamma^I_{RS}\Gamma^S_{KL}\Gamma^R_{MN}G^{JN}-2\Gamma^I_{KS,L}\Gamma^J_{MR}G^{RS} \\
        &\dps\qquad\qquad
        \left.+5\Gamma^I_{KL,S}\Gamma^J_{RM}G^{RS}
        +2\Gamma^I_{KS}\Gamma^S_{LR}(\Gamma^R_{MN}G^{JN}-4\Gamma^J_{MN}G^{RN})\right)\varpi_I\varpi_J\dphi^K\dphi^L\dphi^M \\
        &\dps\qquad
        \left.+\frac{1}{6}\left(V_{I;JK}+3\Gamma^S_{IJ}V_{K;S}+(\Gamma^S_{IJ,K}+\Gamma^R_{IJ}\Gamma^S_{KR})V_S\right)\dphi^I\dphi^J\dphi^K\right],
    } \\
    & \bce{
    	\dps
	{C_i^\phi}^{(0)} & = 0, \\
	\dps
	{C_i^\phi}^{(1)} &
	\dps
	=\varpi_I\partial_i\dphi^I, \\
	\dps
	{C_i^\phi}^{(2)} &
	\dps
	=\dpi_I\partial_i\dphi^I=(\Dpi_I+\Gamma_{IJ}^K\varpi_K\dphi^J)\partial_i\dphi^I, \\
	\dps
	{C_i^\phi}^{(3)} & = 0,
    } \\
    & \bce{
    	\dps
	{C_i^G}^{(0)} & = 0, \\
	\dps
	{C_i^G}^{(1)} & 
	\dps
	= -\frac{2a^3\Mpl^2\calH}{\calN_\IR^2}\alpha_i, \\
	\dps
	{C_i^G}^{(2)} & 
	\dps
	= \frac{4a^3\Mpl^2\calH}{\calN_\IR^3}\alpha\alpha_i+\frac{a\Mpl^2}{\calN_\IR^2}(\alpha_i\nabla^2\psi-\calN_\IR\alpha_j\psi_{ij}), \\
	\dps
	{C_i^G}^{(3)} &
	\dps
	= -\frac{6a^3\Mpl^2\calH}{\calN_\IR^4}\alpha^2\alpha_i -\frac{2a\Mpl^2}{\calN_\IR^3}\alpha(\alpha_i\nabla^2\psi-\alpha_j\psi_{ij}).
    }
}
Therefore one obtains the action order by order as
\bae{
	S^{(0)} =& \int\dd^4x\left[(\pi^{ij}\dot{\gamma}_{ij}-\calN C^G)^{(0)}+(\pi_I\dot{\phi}^I)^{(0)}-\calN^{(0)}{C^\phi}^{(0)}
	-{\beta^i}^{(0)}({C_i^\phi}^{(0)}+{C_i^G}^{(0)})\right]
	\nonumber \\
	=&\int\dd^4x\left[\varpi_I\dot{\varphi}^I-a^3\calN_\IR\left(\frac{1}{2a^6}\varpi_I\varpi^I+V+\frac{3\Mpl^2\calH^2}{\calN_\IR^2}\right)\right], \\
	S^{(1)} =& \int\dd^4x\left[(\pi^{ij}\dot{\gamma}_{ij}-\calN C^G)^{(1)}+(\pi_I\dot{\phi}^I)^{(1)}-\calN^{(0)}{C^\phi}^{(1)}-\calN^{(1)}{C^\phi}^{(0)} \right. \nonumber \\
	&\qquad \left.-{\beta^i}^{(0)}({C_i^\phi}^{(1)}+{C_i^G}^{(1)})-{\beta^i}^{(1)}({C_i^\phi}^{(0)}+{C_i^G}^{(0)})\right] \nonumber \\
	=&\int\dd^4x\left[\Dpi_I\left(\dot{\varphi}^I-\frac{\calN_\IR}{a^3}\varpi^I\right)-\dphi^I\left(D_t\varpi_I+a^3\calN_\IR V_I\right)
	-a^3\alpha\left(\frac{1}{2a^6}\varpi_I\varpi^I+V-\frac{3\Mpl^2\calH^2}{\calN_\IR^2}\right)\right], \\
	S^{(2)}=&\int\dd^4x\left[(\pi^{ij}\dot{\gamma}_{ij}-\calN C^G)^{(2)}+(\pi_I\dot{\phi}^I)^{(2)}-\calN^{(0)}{C^\phi}^{(2)}-\calN^{(1)}{C^\phi}^{(1)} \right. \nonumber \\
	&\qquad \left. -{\beta^i}^{(0)}({C_i^\phi}^{(2)}+{C_i^G}^{(2)})-{\beta^i}^{(1)}({C_i^\phi}^{(1)}+{C_i^G}^{(1)})\right] \nonumber \\
	=&\int\dd^4x\left[\Dpi_ID_t\dphi^I-\frac{3a^3\Mpl^2\calH^2}{\calN_\IR^3}\alpha^2-a^3\alpha\left(\frac{1}{a^6}\varpi^I\Dpi_I+V_I\dphi^I
	+\frac{2\Mpl^2\calH}{\calN_\IR^2}\frac{\nabla^2}{a^2}\psi\right)+\varpi_I\dphi^I\frac{\nabla^2}{a^2}\psi \right. \nonumber \\
	&\qquad -a^3\calN_\IR\left(\frac{1}{2a^6}\Dpi_I\Dpi^I-\frac{1}{2}\dphi_I\frac{\nabla^2}{a^2}\dphi^I+\frac{1}{2}V_{I;J}\dphi^I\dphi^J
	-\frac{1}{2a^6}R_I{}^{JK}{}_L\varpi_J\varpi_K\dphi^I\dphi^L\right) \nonumber \\
	&\qquad -\left(\dot{\varphi}^I-\frac{\calN_\IR}{a^3}\varpi^I\right)\left(\frac{1}{2}(\Gamma^K_{JL,I}+\Gamma^K_{IS}\Gamma^S_{JL})\varpi_K\dphi^J\dphi^L
	+\Gamma_{IJ}^K\dphi^J\Dpi_K\right) \nonumber \\
	&\qquad \left.-\frac{1}{2}\left(D_t\varpi_I+a^3\calN_\IR V_I\right)\Gamma^I_{JK}\dphi^J\dphi^K\right], \\
	S^{(3)}=&\int\dd^4x\left[(\pi^{ij}\dot{\gamma}_{ij}-\calN C^G)^{(3)}+(\pi_I\dot{\phi}^I)^{(3)}-\calN^{(0)}{C^\phi}^{(3)}-\calN^{(1)}{C^\phi}^{(2)} \right. \nonumber \\
	&\qquad \left. -{\beta^i}^{(0)}({C_i^\phi}^{(3)}+{C_i^G}^{(3)})-{\beta^i}^{(1)}({C_i^\phi}^{(2)}+{C_i^G}^{(2)})\right] \nonumber \\
	=&\int\dd^4x\left[\frac{3a^3\Mpl^2\calH^2}{\calN_\IR^4}\alpha^3-\frac{\Mpl^2}{2a\calN_\IR^2}\alpha[(\nabla^2\psi)^2-\psi_{ij}\psi_{ij}] \right. \nonumber \\
	&\qquad-a^3\calN_\IR\left[\frac{1}{a^6}\Gamma^I_{JK}\dphi^K\partial_i\dphi_I\partial_i\dphi^K-\frac{1}{a^6}\Gamma^I_{JK}\Dpi_I\Dpi^J\dphi^K \right. \nonumber \\
    &\quad\qquad
    -\frac{1}{a^6}\left(R_K{}^{IJ}{}_L+\Gamma^I_{KL,R}G^{JR}-\Gamma^I_{LS}\Gamma^S_{KR}G^{JR}+\Gamma^I_{RS}\Gamma^S_{KL}G^{JR}\right)\varpi_I\Dpi_J\dphi^K\dphi^L \nonumber \\
    &\qquad\qquad-\frac{1}{6a^6}\left(R_K{}^{IJ}{}_{L;M}+2\Gamma^S_{KL}R_S{}^{IJ}{}_M+4\Gamma^I_{SK}R_L{}^{JS}{}_M+\Gamma^I_{KL,MS}G^{JS} \right. \nonumber \\
    &\qquad\qquad\qquad+\Gamma^I_{KL,S}(5\Gamma^J_{MR}G^{RS}-\Gamma^S_{MR}G^{JR})
    -2(\Gamma^I_{KS,M}\Gamma^S_{LR}+\Gamma^I_{KS}\Gamma^S_{LR,M})G^{JR} \nonumber \\
    &\qquad\qquad\qquad
    +(\Gamma^I_{RS,K}\Gamma^S_{LM}+\Gamma^I_{RS}\Gamma^S_{LM,K})G^{JR}-\Gamma^I_{RS}\Gamma^S_{KL}\Gamma^R_{MN}G^{JN}-2\Gamma^I_{KS,L}\Gamma^J_{MR}G^{RS} \nonumber \\
    &\qquad\qquad\qquad
    \left.+5\Gamma^I_{KL,S}\Gamma^J_{RM}G^{RS}
    +2\Gamma^I_{KS}\Gamma^S_{LR}(\Gamma^R_{MN}G^{JN}-4\Gamma^J_{MN}G^{RN})\right)\varpi_I\varpi_J\dphi^K\dphi^L\dphi^M \nonumber \\
    &\qquad\qquad
    \left.+\frac{1}{6}\left(V_{I;JK}+3\Gamma^S_{IJ}V_{K;S}+(\Gamma^S_{IJ,K}+\Gamma^R_{IJ}\Gamma^S_{KR})V_S\right)\dphi^I\dphi^J\dphi^K\right]
	\nonumber \\
	&\qquad-a^3\alpha\left[\frac{1}{2a^6}\Dpi_I\Dpi^I+\frac{1}{2a^2}\partial_i\dphi^I\partial_i\dphi_I+\frac{1}{2}V_{I;J}\dphi^I\dphi^J
    -\frac{1}{2a^6}R_I{}^{JK}{}_L\varpi_J\varpi_K\dphi^I\dphi^L \right. \nonumber \\
    &\qquad\qquad
    \left.-\frac{1}{a^6}\Gamma^K_{IJ}\varpi^I\dphi^J\Dpi_K-\frac{1}{2a^6}(\Gamma^I_{KL,J}+\Gamma^I_{JS}\Gamma^S_{KL})\varpi_I\varpi^J\dphi^K\dphi^L
    +\frac{1}{2}\Gamma^K_{IJ}V_K\dphi^I\dphi^J\right] \nonumber \\
	&\left.-\frac{\psi_i}{a^2}\left[(\Dpi_I+\Gamma^K_{IJ}\varpi_K\dphi^J)\partial_i\dphi^I+\frac{4a^3\Mpl^2\calH}{\calN_\IR^3}\alpha\alpha_i+\frac{a\Mpl^2}{\calN_\IR^2}(\alpha_i\nabla^2\psi-\calN_\IR\alpha_j\psi_{ij})\right]\right].
}

To be compared with the standard perturbation theory, we hereafter impose the classical background EoM \YT{though it might not be good in the stochastic approach}
\bae{
    \dot{\varphi}^I=\frac{\calN_\IR}{a^3}\varpi^I, \qquad D_t\varpi_I=-a^3\calN_\IR V_I.
}
Note that $S^{(3)}$ is not covariant. This is because $\dphi^I$ and $\Dpi_I$ are not covariant beyond the linear order. That is, the true cubic action has contributions from $S^{(2)}$ with quadratic expansions of $\dphi^I$ and $\Dpi_I$.
We then express the expansion in the covariant perturbations $Q^I$ and $\tilde{P}_I$ by square parentheses $[1]$, $[2]$, $[3]$, compared to the round parentheses $(1)$, $(2)$, $(3)$ expressing the expansion in $\dphi^I$ and $\Dpi_I$. For example,
\bae{
    \bce{
        \dps
        {\dphi^I}^{[1]}=Q^I, \qquad {\dphi^I}^{[2]}=-\frac{1}{2}\Gamma^I_{JK}Q^JQ^K, \\
        \dps
        \Dpi_I^{[1]}=\tilde{P}_I, \qquad
        \Dpi_I^{[2]}=\Gamma_{IJ}^K\tilde{P}_KQ^J+\frac{1}{2}(\Gamma^S_{IJ,K}+\Gamma^R_{IJ}\Gamma^S_{RK})\varpi_SQ^JQ^K,
    }
}
in our definition.





%\acknowledgments





%\appendix







\bibliography{main}
\end{document}