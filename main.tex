% !TEX TS-program = xelatex%

\documentclass[aps, prd
, preprint
%, twocolumn
, nofootinbib 
, notitlepage
%, superscriptaddress
, longbibliography
]{revtex4-1}
\usepackage{graphicx}
\usepackage[caption=false]{subfig}
\usepackage{mathrsfs}
\usepackage{amsmath,amssymb}
\usepackage{bm}
\usepackage{braket}
\usepackage{listings}
\usepackage{cases}
\usepackage{comment}
\usepackage{soul}
\usepackage{cancel}
\usepackage{cases}
\usepackage[utf8]{inputenc}
\usepackage{url}
\usepackage{longtable}
\usepackage[normalem]{ulem}
\usepackage[colorlinks=true
,urlcolor=blue
,anchorcolor=blue
,citecolor=blue
,filecolor=blue
,linkcolor=blue
,menucolor=blue
%,pagecolor=blue
,linktocpage=true
,pdfproducer=medialab
,pdfa=true
]{hyperref}

%\usepackage{mathpazo}
%\usepackage[no-math]{fontspec}
%\setmainfont{Palatino}
%\setsansfont{Optima}

\newcommand{\dif}[2]{\frac{\mathrm{d} #1}{\mathrm{d} #2}}
\newcommand{\pdif}[2]{\frac{\partial #1}{\partial #2}}
\newcommand{\var}[2]{\frac{\delta #1}{\delta #2}}
\newcommand{\dd}{\mathrm{d}}
\newcommand{\DD}{\mathscr{D}}
\newcommand{\ee}{\mathrm{e}}
\newcommand{\diag}{\mathrm{diag}}
\newcommand{\sgn}{\mathrm{sgn}}
\newcommand{\Mpl}{M_\text{Pl}}
\newcommand{\ns}{n_{{}_\mathrm{S}}}
\newcommand{\cs}{c_{{}_\mathrm{S}}}
\newcommand{\IR}{\text{IR}}
\newcommand{\UV}{\text{UV}}
\renewcommand{\Re}{\mathrm{Re}}
\renewcommand{\Im}{\mathrm{Im}}
\newcommand{\dk}{\frac{\dd^3k}{(2\pi)^3}}
\newcommand{\bbalpha}{{\alpha\!\!\!\alpha}}
\newcommand{\dps}{\displaystyle}
\newcommand{\SIA}{S_\text{IA}}
\newcommand{\eff}{\text{eff}}
\newcommand{\kdx}{\mathbf{k}\cdot\mathbf{x}}

\newcommand{\calD}{\mathcal{D}}
\newcommand{\scrD}{\mathscr{D}}
\newcommand{\calg}{\mathcal{g}}
\newcommand{\calH}{\mathcal{H}}
\newcommand{\scrH}{\mathscr{H}}
\newcommand{\uI}{\text{I}}
\newcommand{\calJ}{\mathcal{J}}
\newcommand{\scrJ}{\mathscr{J}}
\newcommand{\calL}{\mathcal{L}}
\newcommand{\scrL}{\mathscr{L}}
\newcommand{\calN}{\mathcal{N}}
\newcommand{\calO}{\mathcal{O}}
\newcommand{\scrO}{\mathscr{O}}
\newcommand{\calP}{\mathcal{P}}
\newcommand{\calR}{\mathcal{R}}
\newcommand{\uR}{\text{R}}

\newcommand{\bae}[1]{\begin{align} #1 \end{align}}
\newcommand{\bce}[1]{\begin{cases} #1 \end{cases}}
\newcommand{\bfe}[4]{
\begin{figure} 
	\centering
	\includegraphics[#1]{#2}
	\caption{#3}
	\label{#4}
\end{figure}}
\newcommand{\bpme}[1]{\begin{pmatrix} #1 \end{pmatrix}}

\newcommand{\Red}[1]{\textcolor{red}{\sffamily #1}}
\newcommand{\Mag}[1]{\textcolor{magenta}{\sffamily #1}}
\newcommand{\Blue}[1]{\textcolor{blue}{\sffamily #1}}
\newcommand{\mathblue}[1]{\textcolor{blue}{#1}}



\begin{document}
\title{Covariant Fokker-Planck Equation}
\date{\today}

\author{Yuichiro Tada}
\email{tada.yuichiro@e.mbox.nagoya-u.ac.jp}
\affiliation{Department of Physics, Nagoya University, Nagoya 464-8602, Japan}
%\affiliation{Institut d'Astrophysique de Paris, GReCO, UMR 7095 du CNRS et Sorbonne Universit\'e, 
%98bis boulevard Arago, 75014 Paris, France}


\begin{abstract}
Recently Kitamoto pointed out that the one-point quadratic UV terms caused by non-parallel transportation of the vielbein is non-negligible as a new source of the drift term in the Langevin equation,
resulting in a covariant Fokker-Planck equation in the non-linear sigma model~\cite{Kitamoto:2018dek}.
We are trying to summarize all our knowledge, respecting this article.
\end{abstract}

\maketitle
%\tableofcontents



\section{Field-space formulation}

\subsection{spin connection}

According to Ref.~\cite{Kitamoto:2018dek}, the local-frame fields projected by the vielbein seem more
important than the original fields themselves. Therefore let us first derive EoM in terms of the local frame.
The original Hamiltonian equations without metric perturbations are given by
\bae{
    \bce{
        \dps
        \partial_N\phi^I=G^{IJ}\frac{\pi_J}{H}, \\
        \dps
        \partial_N\pi_I=-3\pi_I-\frac{V_I}{H}+\frac{1}{a^2H}\partial_i\left(G_{IJ}\partial_i\phi^J\right)
        -\frac{1}{2a^2H}G_{JK,I}\partial_i\phi^J\partial_i\phi^K-\frac{1}{2H}G^{JK}{}_{,I}\pi_J\pi_K.
    }
}
They can be summarized in a covariant way as
\bae{
    \partial_N\phi^I=G^{IJ}\frac{\pi_J}{H}, \qquad D_N\pi_I=-3\pi_I-\frac{V_I}{H}+\frac{1}{a^2H}D_i\left(G_{IJ}\partial_i\phi^J\right),
}
with use of the covariant spacetime derivatives,
\bae{
    D_N\pi_I=\partial_N\pi_I-\Gamma^K_{IJ}\pi_K\partial_N\phi^J, \qquad 
    D_i\calO_I=\partial_i\calO^I-\Gamma^K_{IJ}\calO_K\partial_i\phi^J.
} 

\newpage

For a function only of $\phi$, these derivatives can be clearly rewritten by the covariant field derivative $\nabla_I$ as
\bae{
    D_{N,i}\calO=(\partial_{N,i}\phi^I)\nabla_I\calO.
}
Therefore, introducing the spin connection
\bae{
    \omega_{Iab}&=e^J_a\nabla_Ie_{Jb} \nonumber \\
    &=\frac{1}{2}e^J_a\left(\partial_Ie_{Jb}-\partial_Je_{Ib}\right)
    -\frac{1}{2}e^I_b\left(\partial_Ie_{Ja}-\partial_Je_{Ia}\right)
    -\frac{1}{2}e^J_ae^K_b(\partial_Je_{Kc}-\partial_Ke_{Jc})e_{Ic},
}
the Hamiltonian equations can be projected to the local frame:
\begin{subnumcases}{}
    \dps
    (\partial_N\phi)_a=\frac{\pi_a}{H}, \\
    \dps
    \partial_N\pi_a=-3\pi_a-\frac{V_a}{H}+\frac{1}{a^2H}\partial_i(\partial_i\phi)_a
    -\frac{1}{H}e^J_c\omega_{Jab}\left(\pi_b\pi_c-\frac{1}{a^2}(\partial_i\phi)_b(\partial_i\phi)_c\right),
    \label{eq: pia EoM}
\end{subnumcases}
where\footnote{Note that field-space vectors are not $\phi^I$ itself but its derivatives $\partial_{N,i}\phi^I$ and that is why the $\phi$'s projection is defined in the way of Eq.~(\ref{eq: def of projection}).}
\bae{\label{eq: def of projection}
    (\partial_{N,i}\phi)_a=e_{Ia}\partial_{N,i}\phi^I, \qquad \pi_a=e^I_a\pi_I, \qquad V_a=e^I_aV_I.
}

While the third term in the right side of Eq.~(\ref{eq: pia EoM}) can be omitted in the long-wavelength limit, the last term gives a non-negligible contribution, according to Ref.~\cite{Kitamoto:2018dek}, 
and such an one-point correlator is estimated as
\bae{
    \pi_a\pi_b-\frac{1}{a^2}(\partial_i\phi)_a(\partial_i\phi)_b\simeq 3H^2g_{ac}g_{bc}=:3H^2g^2_{ab}.
}
Here $g_{ab}$ represents the mode function of the local-frame UV fields. 
The power spectrum of the original fields can be expressed by this one as 
$\calP_{\phi^I\phi^J}=g_{ab}e^I_bg_{ac}e^J_c=g^2_{bc}e^I_be^J_c$. 
If the mode mixing due to the effective mass can be neglected, i.e. $g^2_{ab}\simeq\left(\frac{H}{2\pi}\right)^2\delta_{ab}$, the power spectrum reduces to the standard one $\calP_{\phi^I\phi^J}\simeq\left(\frac{H}{2\pi}\right)^2G^{IJ}$.
Including the contribution of this term, the slow-roll field-space Langevin eq. for IR fields $\varphi^I$ then reads
\bae{\label{eq: Kitamoto Langevin}
    \dd\varphi^I=-\left(\frac{V^I}{3H^2}+\frac{1}{2}g^2_{ab}e^J_b\nabla_Je^I_a\right)\dd N
    +g_{ab}e^I_b\circ\dd W_a.
}
Note the following identical deformation.
\bae{
    e^I_ag^2_{bc}e^J_c\omega_{Jab}=e^I_ag^2_{bc}e^J_ce^K_a\nabla_Je_{Kb}=g^2_{bc}e^J_cG^{IK}\nabla_Je_{Kb}=g^2_{bc}e^J_c\nabla_Je^I_c.
}

This new drift term represents non-uniformity of the local frame. If the field space is exactly flat,
the vielbein can be taken so that its covariant derivative vanishes at all field space points.
Also if one considers only deterministic background trajectory $\phi^I_0$ without stochastic noise,
there is a useful choice of the vielbein satisfying the parallel transported condition $\dot{\phi}_0^I\nabla_Ie^J_a=0$. 
However in the stochastic cases, one must consider all field space points in principle and therefore this spin connection should be taken into account in general.


\subsection{It\^o-Stratonovich stuff and covariant Fokker-Planck}

In the Langevin eq.~(\ref{eq: Kitamoto Langevin}), we adopted the Stratonovich noise integral, reflecting the fact that the Stratonovich works for metric type interactions.
On the other hand, one has to keep in mind that the potential type interaction should be treated in the It\^o's way.
The potential interaction, or the effective mass correction in other words, affects the local-frame amplitude $g_{ab}$.
Following the Tokuda's approach, one then introduces dummy fields $\bar{\varphi}^I$ representing slightly past field values (or fields with a slightly large coarse-graining scale) and $g_{ab}$ is assumed to be a function of $\bar{\varphi}^I$.
After moving to the It\^o's expression, noting that functions of $\bar{\varphi}^I$ are not affected by differentiations in $\phi^I$, one can take a limit of $\bar{\varphi}^I\to\varphi^I$:
\bae{
    \dd\varphi^I&=-\left(\frac{V^I}{3H^2}+\frac{1}{2}g^2_{ab}(\bar{\varphi})e^J_b\nabla_Je^I_a\right)\dd N
    +g_{ab}(\bar{\varphi})e^I_b\circ\dd W_a \nonumber \\
    &=-\left(\frac{V^I}{3H^2}+\frac{1}{2}g^2_{ab}(\bar{\varphi})e^J_b\nabla_Je^I_a-\frac{1}{2}g_{ab}(\bar{\varphi})e^J_b\partial_J(g_{ac}(\bar{\varphi})e^I_c)\right)\dd N+g_{ab}(\bar{\varphi})e^I_b\dd W_a \nonumber \\
    &\hspace{-5pt}\underset{\bar{\varphi}\to\varphi}{=}-\left(\frac{V^I}{3H^2}+\frac{1}{2}g_{ab}e^J_b(\nabla_J-\partial_J)e^I_a\right)\dd N+g_{ab}e^I_b\dd W_a \nonumber \\
    &=-\left(\frac{V^I}{3H^2}+\frac{1}{2}\Gamma^I_{JK}\calP_{\phi^J\phi^K}\right)\dd N+g_{ab}e^I_b\dd W_a.
}
The corresponding Fokker-Planck equation reads
\bae{
    \partial_NP&=\partial_I\left[\left(\frac{V^I}{3H^2}+\frac{1}{2}\Gamma^I_{JK}\calP_{\phi^J\phi^K}\right)P\right]+\frac{1}{2}\partial_I\partial_J(\calP_{\phi^I\phi^J}P) \nonumber \\
    &=\partial_I\left[\left(\frac{V^I}{3H^2}+g_a^J\nabla_Jg^I_a\right)P\right]+\frac{1}{2}\partial_I\left[g^I_a\partial_J(g^J_aP)\right],
}
where $g^I_a=g_{ab}e^I_b$. Making use of the formula $\frac{1}{\sqrt{G}}\nabla_I(\sqrt{G}\calO^I)=\nabla_I\calO^I$, it can be easily covariantized as
\bae{
    \partial_NP_s&=\nabla_I\left[\left(\frac{V^I}{3H^2}+g^J_a\nabla_Jg^I_a\right)P_s\right]+\frac{1}{2}\nabla_I\left[g^I_a\nabla_J(g^J_aP_s)\right] \nonumber \\
    &=\nabla_I\left(\frac{V^I}{3H^2}P_s\right)+\frac{1}{2}\nabla_I\nabla_J(\calP_{\phi^I\phi^J}P_s),
}
for the field-space scalar PDF $P_s=P/\sqrt{G}$. Therefore the resultant Fokker-Planck equation is successfully field-space covariant and independent of the choice of the vielbein.






%\acknowledgments





%\appendix







\bibliography{main}
\end{document}